\section*{Prior Work}

Many aspects of GNSS technology are open. The signals themselves are
defined in public technical documents. File formats associated with
observations and ephemerides are open as well. The case is different
for GNSS processing software. Commercial software packages, often
written by receiver manufacturers, are proprietary by nature.  Others
packages offered freely must be licensed to a site or an
individual. Often \mbox{MATLAB \textregistered} scripts that are
publicly shared are accompanied by a license restriction prohibiting
their redistribution or modification.

In contrast, the GPSTk project source code is free to use, modify, and
redistribute, per the terms of it license, the GNU Lesser General Public License
(LGPL) version 2.1. The GPSTk is not alone in providing an open source suite to
the GNSS research and development community. Several other projects
offer software under the terms of the LGPL, the GPL or other open
source licenses.

The focus of the OpenSourceGPS project is to create open source
receivers~\cite{osgpsfirstpaper}. The first hardware targeted by the
project were PC cards controlled using the Mitel/Plessy/Zarlink chip set. 
Because the chip set is no longer manufactured, the company GPS Creations 
has partnered with the project to new cards. In recent years, OpenSourceGPS and GPS
Creations have focused on developing a narrow-band software
receiver~\cite{osgpsemulator,osgpsfullswrx}.

In contrast to targeting receiver internals, the bulk of open source
projects target the external interfaces of commercial units. Two
notable projects are GPSBabel and
gpsd~\cite{gpsbabelwebsite,gpsdwebsite}. GPSBabel translates way-points
so they can be moved among receiver manufacturers. The gpsd provides a
network service that allows a personal computer to communicate
real-time, differential corrections over a network.

