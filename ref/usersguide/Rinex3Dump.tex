%\documentclass{article}
%\usepackage{fancyvrb}
%\usepackage{src/perltex}
%\usepackage{src/xcolor}
%\usepackage{listings}
%\usepackage{longtable}
%\usepackage{multirow}
%\RecustomVerbatimEnvironment{Verbatim}{Verbatim}{frame=single}

\newcommand{\outputsize}{footnotesize}

\perlnewcommand{\getuse}[1]
{
        my $command = $_[0];
        $command = $command." > temp 2>&1";
        system("$command");

	my $output = "";
        open(input,"temp");
        while(my $line = <input>){$output = $output.$line;}
        close(input);

        return  "\\begin{\\outputsize}\n" . "\\begin{Verbatim}\n" .
                $output .
                "\\end{Verbatim}\n" . "\\end{\\outputsize}\n";
}

\perlnewcommand{\printexec}[1]
{
	my $exec = $_[0];

	return "\\begin{\\outputsize}\n" . "\\begin{Verbatim}\n" .
		$exec . "\n" .
		"\\end{Verbatim}\n" . "\\end{\\outputsize}\n";
}

\perlnewcommand{\application}[1]
{
	my $app = $_[0];
	return "\\emph{" . $app . "}";
}

%\begin{document}

\index{Rinex3Dump!application writeup}
\section{\emph{Rinex3Dump}}
\subsection{Overview}
The application reads a RINEX3 file and dumps the obervation data for the given satellite(s) to the standard output.

\subsection{Usage}
\begin{\outputsize}
\begin{longtable}{lll}
\multicolumn{3}{c}{\application{Rinex3Dump}} \\
\multicolumn{3}{l}{\textbf{Optional Arguments}} \\
\entry{-f}{--file $<$file$>$}{Input file is a RINEX observation file. This option may be repeated.
	Optional, but may be needed in case of ambiguity.}{3}
\entry{}{--format $<$format$>$}{The format of the time output. Default is \%4F \%10.3g.}{2}
\entry{-h}{--help}{Prints out this help and exits.}{1}
\entry{-n}{--num}{Make output purely numeric, ie. no header, no system char on satellites.}{2}
\entry{-o}{--obs $<$obs$>$}{RINEX observation type (eg. C1C) found in the file header.
	Optional, but may be needed in case of ambiguity.}{3}
\entry{-p}{--pos}{Only output positions from aux headers, ie. sat and obs are ignored.}{2}
\entry{-s}{--sat $<$sat$>$}{RINEX satellite ID (eg. For GPS PRN 31, $<$sat$>$ = G01).
	Optional, but may be needed in case of ambiguity.}{3}
\entry{-v}{--verbose}{Prints out verbose output.}{1}
\end{longtable}
\begin{verbatim}
Rinex3Dump usage: Rinex3Dump [-n] <rinex obs file> [<satellite(s)> <obstype(s)>] 

The optional argument -n tells Rinex3Dump its output should be purely numeric.
\end{verbatim}
\end{\outputsize}

\subsection{Examples}
\begin{\outputsize}
\begin{lstlisting}
> Rinex3Dump algo1580.06o 3 4 5

# Rinex3Dump file: algo1580.06o Satellites: G03 G04 G05 Observations: ALL
# Week  GPS_sow Sat            L1 L S            L2 L S            C1 L S
1378 259200.000 G03  -3843024.647 0 3  -2994560.443 0 1  23796436.087 0 0
1378 259230.000 G03  -3954052.735 0 3  -3081075.654 0 2  23775308.750 0 0
1378 259260.000 G03  -4064994.465 0 2  -3167523.561 0 3  23754197.617 0 0
1378 259290.000 G03  -4175846.973 0 3  -3253901.944 0 3  23733104.211 0 0
1378 259320.000 G03  -4286607.460 0 4  -3340208.647 0 3  23712026.249 0 0
1378 259350.000 G03  -4397272.869 0 4  -3426441.227 0 3  23690967.159 0 0

 . . .

          P2 L S            P1 L S            S1 L S            S2 L S
23796439.457 0 0  23796436.350 0 0        21.100 0 0        11.000 0 0
23775311.168 0 0  23775308.182 0 0        22.100 0 0        17.800 0 0
23754199.648 0 0  23754196.550 0 0        17.000 0 0        18.600 0 0
23733104.928 0 0  23733102.480 0 0        19.900 0 0        21.600 0 0
23712027.682 0 0  23712024.790 0 0        24.200 0 0        19.300 0 0
23690968.861 0 0  23690965.837 0 0        25.600 0 0        19.900 0 0

 . . .
\end{lstlisting}
\end{\outputsize}

\subsection{Notes}
MATLAB and Octave can read the purely numeric output.

%\end{document}
