%\documentclass{article}
%\usepackage{fancyvrb}
%\usepackage{perltex}
%\usepackage{xcolor}
%\usepackage{listings}
%\usepackage{longtable}
%\usepackage{multirow}
%\RecustomVerbatimEnvironment{Verbatim}{Verbatim}{frame=single}

\newcommand{\outputsize}{footnotesize}

\perlnewcommand{\getuse}[1]
{
        my $command = $_[0];
        $command = $command." > temp 2>&1";
        system("$command");

	my $output = "";
        open(input,"temp");
        while(my $line = <input>){$output = $output.$line;}
        close(input);

        return  "\\begin{\\outputsize}\n" . "\\begin{Verbatim}\n" .
                $output .
                "\\end{Verbatim}\n" . "\\end{\\outputsize}\n";
}

\perlnewcommand{\printexec}[1]
{
	my $exec = $_[0];

	return "\\begin{\\outputsize}\n" . "\\begin{Verbatim}\n" .
		$exec . "\n" .
		"\\end{Verbatim}\n" . "\\end{\\outputsize}\n";
}

\perlnewcommand{\application}[1]
{
	my $app = $_[0];
	return "\\emph{" . $app . "}";
}

%\begin{document}


\index{CalcDOPs!application writeup}
\section{\emph{CalcDOPs}}
\subsection{Overview}
This application reads SV almanac data (one file per day of observation) from a FIC, FICA or a RINEX Nav file, then computes and displays visibility information.
Dilution of precision values from that data are calculated using standard methods.  See for example:
\begin{itemize}
\item AIAA GPS Theory and Applications vol. 1, Ed. Parkinson \& Spilker, pp. 414.
\item GPS Signals, Measurements, and Performance, 2ed., Misra \& Enge, pp. 203.
\end{itemize}

\subsection{Usage}
\begin{\outputsize}
\begin{longtable}{lll}
\multicolumn{3}{c}{\application{CalcDOPs}} \\
\multicolumn{3}{l}{\textbf{Required Arguments}} \\
\entry{Short Arg.}{Long Arg.}{Description}{1}
\entry{-i}{$<$inputfile$>$}{Input file for day to be calculated.}{1}
& & \\
\multicolumn{3}{l}{\textbf{Optional Arguments}} \\
\entry{-p $<$inputfile$>$}{}{Input file for previous day (ephemeris mode only).}{2}
\entry{-o $<$outputfile$>$}{}{Grid output file (default DOPs.out).}{1}
\entry{-sf $<$outputfile$>$}{}{Stats output file (default DOPs.stat).}{1}
\entry{-tf $<$outputfile$>$}{}{Time steps output file (default DOPS.times).}{2}
\entry{-l $<$outputfile$>$}{}{Log output file (default DOPS.log).}{1}
\entry{-rs}{}{Read from stats file.}{1}
\entry{-a}{}{Work in almanac mode (ephemeris mode is default).}{2}
\entry{-w -s $<$week$>$ $<$sow$>$}{}{Starting time tag.}{1}
\entry{-x $<$prn$>$}{}{Exclude satellite PRN.}{1}
\entry{-t $<$dt$>$}{}{Time spacing.}{1}
\entry{-na}{}{North America only.}{1}
\entry{-d}{}{Dump grid results at each time step (time-intensive).}{2}
\entry{-h}{--help}{Output options info and exit.}{1}
\entry{-v}{}{Print version info and exit.}{1}
\end{longtable}
\end{\outputsize}

\subsection{Notes}
*\ Abort/failure error codes given on return:
\begin{center}
\begin{tabular}{ll}
   -1  &could not open input data file\\
   -2  &could not identify input data file type\\
   -3  &fewer than 4 satellite almanacs available\\
   -4  &could not allocate GridStats data types\\
   -5  &could not open input  stats file\\
   -6  &could not open output grid  file\\
   -7  &could not open output stats file\\
   -8  &could not open output log   file\\
& \\
& \\
\end{tabular}
\end{center}


 Essential variables not documented below at declaration:
\begin{center}
\begin{tabular}{ll}
   NtrofN &  number of cells/times with $<$ 5 SVs visible during the time period\\
   NpeakH &  number cells/times w/ HDOP $>$ 10\\
   NpeakP &  number cells/times w/ PDOP $>$ 10\\
   IworstN&  index in Grid[] of cell with worst nsvs (number of satellites)\\
   IworstH&  index in Grid[] of cell with worst HDOP\\
   IworstP&  index in Grid[] of cell with worst PDOP\\
   WorstN &  value of nsvs at IworstN\\
   WorstH &  value of HDOP at IworstH\\
   WorstP &  value of PDOP at IworstP\\
   TworstN&  time tag (CommonTime class) of WorstN\\
   TworstH&  time tag (CommonTime class) of WorstH\\
   TworstP&  time tag (CommonTime class) of WorstP\\
& \\
$ \\
\end{tabular}
\end{center}



\begin{enumerate}
\item GPS only, using PRNs hard-wired to SV numbers 1-32.
\item Elevation limit is hard-wired to 5 degrees above horizion.
\item "North America" means the northern half-hemisphere: -180 to 0 deg long.,
      0 to 90N latitude.
\item Ephemeris mode is default, almanac mode is optional.  Ephemeris mode is
      preferred, because it excludes unhealthy satellites for any time when they
      transmitted an unhealthy flag.  Almanac mode will generally not exclude SVs
      when they were unhealthy (typical), or may erroneously exclude them for an
      entire day (rarely).
\item If 2 input files are given, the default start time is midnight on the day
      to be calculated.  A previous-day input file can be given only in ephemeris
      mode, not almanac.
\item The code uses geodetic coordinates for all calculations.
\item The -d option is useful for e.g. making movies of DOPs throughout a day.
\end{enumerate}
%\end{document}

