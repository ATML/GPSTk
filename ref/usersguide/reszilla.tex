%\documentclass{article}
%\usepackage{fancyvrb}
%\usepackage{src/perltex}
%\usepackage{src/xcolor}
%\usepackage{listings}
%\usepackage{multirow}
%\usepackage{longtable}
%\RecustomVerbatimEnvironment{Verbatim}{Verbatim}{frame=single}

\newcommand{\outputsize}{footnotesize}

\perlnewcommand{\getuse}[1]
{
        my $command = $_[0];
        $command = $command." > temp 2>&1";
        system("$command");

	my $output = "";
        open(input,"temp");
        while(my $line = <input>){$output = $output.$line;}
        close(input);

        return  "\\begin{\\outputsize}\n" . "\\begin{Verbatim}\n" .
                $output .
                "\\end{Verbatim}\n" . "\\end{\\outputsize}\n";
}

\perlnewcommand{\printexec}[1]
{
	my $exec = $_[0];

	return "\\begin{\\outputsize}\n" . "\\begin{Verbatim}\n" .
		$exec . "\n" .
		"\\end{Verbatim}\n" . "\\end{\\outputsize}\n";
}

\perlnewcommand{\application}[1]
{
	my $app = $_[0];
	return "\\emph{" . $app . "}";
}

%\begin{document}

\index{reszilla!application writeup}
\section{\emph{reszilla}}
\subsection{Overview}
Reszilla is an application that computes various residuals from GPS
pseudorange, phase and doppler data. These data are often refered to
as raw observations. The two types of residuals that are currently
computed are an Observed Range Deviation (ORD), and a double
difference (DD). Once these residuals are computed, statistical
summaries of these differences are computed and output to the
user. Optionally, the residuals themselves may be output.

\subsection{Observed Range Deviations}
An ORD is basically the observed range to an SV differenced from the
estimated range to that SV. There are many terms that go into
computing the estimated range and/or correcting the observed range for
known effects. When all of these effects are accounted for (as
reszilla is capable of doing) ORDs can be in the 10-30 cm range for a
geodetic quality GPS receiver. Pretty impressive when you consider
that the range to the SV is somewhere between 20 to 26 million meters.

For many GPS receivers, the most significant effect to account for is the receiver
clock offset. This is the difference between the receivers internal time and true
GPS time. This parameter is often computed as part of a PVT solution. This is not
how reszilla works. Reszilla is provided a surveyed position of the receiver
antenna, and it makes a more accurate estimate of the receiver clock offset by
averaging the residuals of all SVs in track.

%ToDo:
%   discuss of the need for an accurate antenna position
%   discuss of the relative errors in the broadcast versus a precise ephemeris
%   discuss the difference between SV time and RX time

The ORD Options:

\begin{verbatim}
 -n, --search-near
 --svtime
 --check-obs
 --omode
\end{verbatim}

\subsection{Double Difference Residuals}
While many double differences exist, reszilla computes an the first
difference to a master SV and the second difference to a second
receiver.  This double difference removes receiver clock error, iono,
trop, and SV clock errors. When the two receivers are connected to a
common antenna (often referred to as a zero-baseline setup) and are of
the same type, even the multipath is differenced out. What is left is
basically receiver tracking noise and receiver tracking errors.

One complicating factor in computing this DD is that while the clock
errors in the receivers cancel out, there is still an error associated
with the motion of the satellite during the interval between when the
two receivers computing their observation. To remove this error, an
estimate of the clock offset between the two receivers is
need. Reszilla can get this estimate in one of two ways; estimates
this by computing a clock estimate for each receiver as described
under the ORD section or reading the estimates from the rinex obs data
files. These two estimates are then differenced to get the offset
between the two receivers.

Another complicating factor is that the phase observations normally
have an "integer ambiguity" associated with them. When the DD phase
observation is computed, it will have the difference between the two
receivers ambiguity. Often this number can be quite big. Removing this
ambiguity is often referred to as debiasing the data. This process
involves much black magic and slight of hand. Do not delve into this
or even look too closely at the details or you will be sullied.

DD Options:

\begin{verbatim}
 --ddmode
 --min-arc-time
 --min-arc-gap
 --min-arg-length
\end{verbatim}

\subsection{Data Input}

Several different types of data are required to compute these
residuals; the raw observations, the receiver antenna position, the
satellite position, and optionally  weather observations. The raw
observations may be supplied to reszilla in one of several formats;
rinex obs (see RinexObsData class), smodf (see SMODFData class), and
MDP (see MDPObsEpoch class in apps/MDPtools). The reciever antenna
postion may be specified in the rinex obs header or via a station
coordinates file (see MSCData class).

Options:

\begin{verbatim}
 -o, --obs1
 -2, --obs2
 -e, --eph
 -w, --weather
 -c, --clock-from-rinex
 -m, --msid
 --msc
\end{verbatim}

\subsection{Output}
There are two general types of output that reszilla produces -
statistical summaries and the raw residuals.  The mean, standard 
deviation, and maximum value of the residuals are calculated 
as a function of specified elevation ranges and are output in a 
statistics table. Looking at the results for each elevation bin 
is useful as ORDs tend to be much a higher when satellites are 
lower on the horizon. For a more thorough analysis, the ORD or DD 
residuals calculated by reszilla may be output in a matrix format 
to a file with columns for time, PRN, elevation, ORD or clock residual, 
IODC, satellite health, and a flag for the residual type.  The flag 
specifies exactly which of the 13 possible residual types the data 
on that row represent, depending on the method used for calculation. 

One benefit of this output feature is that residuals can be looked at 
for particular time periods or PRNs. Fortunately there is a companion 
plotting tool that makes this simple. Given a reszilla output file, 
the dplot program will plot residuals and, if specified, receiver clock 
estimates versus time using gnuplot. A user may specify the time 
range, stripping value, and PRN(s) to use in the plot, as well as a
filename for saving the result. 

Output Options:

\begin{verbatim}
 --keep-unhealthy
 -s, --no-stats
 --cycle-slips
 -r, --raw-output
 -t, --time-format
 --clock-est
 -b, --elev-bin
 --sigma
 -v, --verbosity
\end{verbatim}

%Examples:

 %  < to be added later >
 %  smodf with met and sp3
 %  rinex-30 with met and sp3
 %  rinex-30 without met with sp3
 %  rinex-30 with met and fic
 %  rinex-30 without met with fic

\subsection{Usage}

\begin{\outputsize}
\begin{longtable}{lll}
\multicolumn{3}{l}{\textbf{Required Arguments}} \\
\entry{Short Arg.}{Long Arg.}{Description}{1}
\entry{-o}{--obs1=ARG}{Observation data file name. If this option is specified more than once the contents of all files will be used.}{3}
& & \\
\multicolumn{3}{l}{\textbf{Optional Arguments}} \\
\entry{Short Arg.}{Long Arg.}{Description}{1}
\entry{-h}{--help}{Generates help and usage.}{1}
\entry{-2}{--position=ARG}{Second receiver's observation data file name. Only used when computing a double difference. If this option is specified more than once the contents of all the files will be used.}{4}
\entry{}{--msc=ARG}{Station coordinate file.}{1}
\entry{-e}{--ephemeris=ARG}{Ephemeris data file name (either broadcast in RINEX nav, broadcast in FIC, or precise in SP3).}{3}
\entry{-w}{--weather}{Weather data file name (RINEX met format only).}{2}
\entry{-n}{--search-near}{Use BCEphemeris.searchNear()}{1}
\entry{-c}{--clock-from-rinex}{Use the receiver clock offset from the rinex obs data.}{2}
\entry{}{--svtime}{Observation data is in SV time  frame. The default is RX time frame.}{2}
\entry{}{--check-obs}{Report data rate, order of data, data present, data gaps.}{2}
\entry{}{--keep-unhealthy}{Keeps unhealthy SVs in the statistics, default is to toss.}{2}
\entry{-s}{--no-stats}{Don't compute output the statistics.}{1}
\entry{}{--cycle-slips}{Output a list of cycle slips.}{1}
\entry{-r}{--raw-output=ARG}{Dump the computed residuals/ords into specified file. If '-' is given as the file name, the output is sent to standard output. The default is to not otput the raw residuals.}{4}
\entry{}{--start-time=TIME}{Ignore obs data prior to this time in the analysis. The time is specified using the format \%4Y/\%03j/\%02H:\%02M:\%05.2f. The default value is to start with the first data found.}{4}
\entry{}{--stop-time=TIME}{Ignore obs data after to this time in the analysis. The time is specified using the format \%4Y/\%03j/\%02H:\%02M:\%05.2f. The default value is to process all data.}{4}
\entry{-t}{--time-format=ARG}{Daytime format specifier used for the timestamps in the raw output. The default is "\%Y \%3j \%02H:\%02M:\%04.1f". If this option is specified with the format as "s", the format "\%Y \%3j \%7.1s" is used. If this option is specified with the format as "s", the format "\%Y \%3j \%02H:\%02M:\%02S" is used.}{7}
\entry{}{--omode=ARG}{ORD mode: P1P2, C1P2, C1, P1, P2. The default is p1p2}{2}
\entry{}{--clock-est}{Compute a linear clock estimate.}{1}
\entry{}{--ddmode=ARG}{Double difference residual mode: none, sv, or c1p2. The default is sv.}{2}
\entry{}{--min-arc-time=ARG}{The minimum length of time (in seconds) that a sequence of observations must span to be considered as an arc. The default value is 60.0 seconds.}{4}
\entry{}{--min-arc-gap=ARG}{The minimum length of time (in seconds) between two arcs for them to be considered separate arcs. The default value is 60.0 seconds.}{3}
\entry{}{--min-arc-length=NUM}{The minimum number of epochs that can be considered an arc. The default value is 5 epochs.}{2}
\entry{-b}{--elev-bin=ARG}{A range of elevations, used in computing the statistical summaries. Repeat to specify multiple bins. The default is "-b 0-10 -b 10-20 -b 20-60 -b 10-90".}{4}
\entry{}{--sigma=NUM}{ Multiplier for sigma stripping used in computation of statistics on the raw residuals. The default value is 6.}{3}
\entry{-v}{--verbosity=NUM}{How much detail to provide about intermediate steps.}{2}
\entry{}{0}{nothing but the results}{1}
\entry{}{1}{Output status before potentially time consuming operations (default)}{2}
\entry{}{2}{more details about each step and the options chosen}{2}
\entry{}{3}{add the reasons for editing data}{1}
\entry{}{4}{dump intermediate values for each epoch (can be QUITE verbose)}{1}

\end{longtable}

\begin{verbatim}
Types in the raw output files:
   0 - c1p2 observed range deviation
   50 - computed clock, difference from estimate, strip
   51 - linear clock estimate, abdev
Double difference types:
   10 - c1     20 - c2
   11 - p1     21 - p2
   12 - l1     22 - l2
   13 - d1     23 - d2
   14 - s1     24 - s2
\end{verbatim}
\end{\outputsize}

\subsection{Examples}

\begin{\outputsize}
\begin{lstlisting}
reszilla --omode=p1 --svtime --msc=mscoords.cfg -m 85401 -o asm2004.138 -e s011138a.04n
\end{lstlisting}
\end{\outputsize}

\subsection{Notes}
The criteria min-arc-time and min-arc-length are both required to be met
for a arc to be valid in double difference mode.
All output quantities (stddev, min, max, ord, clock, double differnce, ...)
are in meters.

%\end{document}

