%\documentclass{article}
%\usepackage{fancyvrb}
%\usepackage{src/perltex}
%\usepackage{src/xcolor}
%\usepackage{listings}
%\usepackage{multirow}
%\usepackage{longtable}
%\RecustomVerbatimEnvironment{Verbatim}{Verbatim}{frame=single}

\newcommand{\outputsize}{footnotesize}

\perlnewcommand{\getuse}[1]
{
        my $command = $_[0];
        $command = $command." > temp 2>&1";
        system("$command");

	my $output = "";
        open(input,"temp");
        while(my $line = <input>){$output = $output.$line;}
        close(input);

        return  "\\begin{\\outputsize}\n" . "\\begin{Verbatim}\n" .
                $output .
                "\\end{Verbatim}\n" . "\\end{\\outputsize}\n";
}

\perlnewcommand{\printexec}[1]
{
	my $exec = $_[0];

	return "\\begin{\\outputsize}\n" . "\\begin{Verbatim}\n" .
		$exec . "\n" .
		"\\end{Verbatim}\n" . "\\end{\\outputsize}\n";
}

\perlnewcommand{\application}[1]
{
	my $app = $_[0];
	return "\\emph{" . $app . "}";
}

%\begin{document}

\index{reszilla!application writeup}
\section{\emph{reszilla}}
\subsection{Overview}

\subsection{Usage}
\begin{\outputsize}
\begin{longtable}{lll}
\multicolumn{3}{l}{\textbf{Required Arguments}} \\
\entry{Short Arg.}{Long Arg.}{Description}{1}
\entry{-o}{--obs1=ARG}{Observation data file name. If this option is specified more than once the contents of all files will be used.}{3}
& & \\
\multicolumn{3}{l}{\textbf{Optional Arguments}} \\
\entry{Short Arg.}{Long Arg.}{Description}{1}
\entry{-h}{--help}{Generates help and usage.}{1}
\entry{-2}{--position=ARG}{Second receiver's observation data file name. Only used when computing a double difference. If this option is specified more than once the contents of all the files will be used.}{4}
\entry{}{--msc=ARG}{Station coordinate file.}{1}
\entry{-e}{--ephemeris=ARG}{Ephemeris data file name (either broadcast in RINEX nav, broadcast in FIC, or precise in SP3).}{3}
\entry{-w}{--weather}{Weather data file name (RINEX met format only).}{2}
\entry{-n}{--search-near}{Use BCEphemeris.searchNear()}{1}
\entry{-c}{--clock-from-rinex}{Use the receiver clock offset from the rinex obs data.}{2}
\entry{}{--svtime}{Observation data is in SV time  frame. The default is RX time frame.}{2}
\entry{}{--check-obs}{Report data rate, order of data, data present, data gaps.}{2}
\entry{}{--keep-unhealthy}{Keeps unhealthy SVs in the statistics, default is to toss.}{2}
\entry{-s}{--no-stats}{Don't compute \& output the statistics.}{1}
\entry{}{--cycle-slips}{Output a list of cycle slips.}{1}
\entry{-r}{--raw-output=ARG}{Dump the computed residuals/ords into specified file. If '-' is given as the file name, the output is sent to standard output. The default is to not otput the raw residuals.}{4}
\entry{}{--start-time=TIME}{Ignore obs data prior to this time in the analysis. The time is specified using the format \%4Y/\%03j/\%02H:\%02M:\%05.2f. The default value is to start with the first data found.}{4}
\entry{}{--stop-time=TIME}{Ignore obs data after to this time in the analysis. The time is specified using the format \%4Y/\%03j/\%02H:\%02M:\%05.2f. The default value is to process all data.}{4}
\entry{-t}{--time-format=ARG}{Daytime format specifier used for the timestamps in the raw output. The default is "\%Y \%3j \%02H:\%02M:\%04.1f". If this option is specified with the format as "s", the format "\%Y \%3j \%7.1s" is used. If this option is specified with the format as "s", the format "\%Y \%3j \%02H:\%02M:\%02S" is used.}{7}
\entry{}{--omode=ARG}{ORD mode: P1P2, C1P2, C1, P1, P2. The default is p1p2}{2}
\entry{}{--clock-est}{Compute a linear clock estimate.}{1}
\entry{}{--ddmode=ARG}{Double difference residual mode: none, sv, or c1p2. The default is sv.}{2}
\entry{}{--min-arc-time=ARG}{The minimum length of time (in seconds) that a sequence of observations must span to be considered as an arc. The default value is 60.0 seconds.}{4}
\entry{}{--min-arc-gap=ARG}{The minimum length of time (in seconds) between two arcs for them to be considered separate arcs. The default value is 60.0 seconds.}{3}
\entry{}{--min-arc-length=NUM}{The minimum number of epochs that can be considered an arc. The default value is 5 epochs.}{2}
\entry{-b}{--elev-bin=ARG}{A range of elevations, used in computing the statistical summaries. Repeat to specify multiple bins. The default is "-b 0-10 -b 10-20 -b 20-60 -b 10-90".}{4}
\entry{}{--sigma=NUM}{ Multiplier for sigma stripping used in computation of statistics on the raw residuals. The default value is 6.}{3}
\entry{-v}{--verbosity=NUM}{How much detail to provide about intermediate steps.}{2}
\entry{}{0}{nothing but the results}{1}
\entry{}{1}{Output status before potentially time consuming operations (default)}{2}
\entry{}{2}{more details about each step and the options chosen}{2}
\entry{}{3}{add the reasons for editing data}{1}
\entry{}{4}{dump intermediate values for each epoch (can be QUITE verbose)}{1}

\end{longtable}
\begin{verbatim}
Types in the raw output files:
   0 - c1p2 observed range deviation
   50 - computed clock, difference from estimate, strip
   51 - linear clock estimate, abdev
Double difference types:
   10 - c1     20 - c2
   11 - p1     21 - p2
   12 - l1     22 - l2
   13 - d1     23 - d2
   14 - s1     24 - s2
\end{verbatim}
\end{\outputsize}

\subsection{Examples}

\begin{\outputsize}
\begin{lstlisting}
reszilla --omode=p1 --svtime --msc=mscoords.cfg -m 85401 -o asm2004.138 -e s011138a.04n
\end{lstlisting}
\end{\outputsize}

\subsection{Notes}
The criteria min-arc-time and min-arc-length are both required to be met
for a arc to be valid in double difference mode.
All output quantities (stddev, min, max, ord, clock, double differnce, ...)
are in meters.

%\end{document}

