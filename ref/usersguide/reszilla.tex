%\documentclass{article}
%\usepackage{fancyvrb}
%\usepackage{perltex}
%\usepackage{xcolor}
%\usepackage{listings}
%\usepackage{multirow}
%\usepackage{longtable}
%\RecustomVerbatimEnvironment{Verbatim}{Verbatim}{frame=single}

\newcommand{\outputsize}{footnotesize}

\perlnewcommand{\getuse}[1]
{
        my $command = $_[0];
        $command = $command." > temp 2>&1";
        system("$command");

	my $output = "";
        open(input,"temp");
        while(my $line = <input>){$output = $output.$line;}
        close(input);

        return  "\\begin{\\outputsize}\n" . "\\begin{Verbatim}\n" .
                $output .
                "\\end{Verbatim}\n" . "\\end{\\outputsize}\n";
}

\perlnewcommand{\printexec}[1]
{
	my $exec = $_[0];

	return "\\begin{\\outputsize}\n" . "\\begin{Verbatim}\n" .
		$exec . "\n" .
		"\\end{Verbatim}\n" . "\\end{\\outputsize}\n";
}

\perlnewcommand{\application}[1]
{
	my $app = $_[0];
	return "\\emph{" . $app . "}";
}

%\begin{document}

\index{reszilla!application writeup}
\section{\emph{reszilla}}
\subsection{Overview}
Reszilla is an application that computes various residuals from GPS
pseudorange, phase and doppler data. These data are often refered to
as raw observations. The two types of residuals that are currently
computed are an Observed Range Deviation (ORD), and a double
difference (DD). Once these residuals are computed, statistical
summaries of these differences are computed and output to the
user. Optionally, the residuals themselves may be output.

\subsection{Observed Range Deviations}
An ORD is basically the observed range to an SV differenced from the
estimated range to that SV. There are many terms that go into
computing the estimated range and/or correcting the observed range for
known effects. When all of these effects are accounted for (as
reszilla is capable of doing) ORDs can be in the 10-30 cm range for a
geodetic quality GPS receiver. Pretty impressive when you consider
that the range to the SV is somewhere between 20 to 26 million meters.

For many GPS receivers, the most significant effect to account for is the receiver
clock offset. This is the difference between the receivers internal time and true
GPS time. This parameter is often computed as part of a PVT solution. This is not
how reszilla works. Reszilla is provided a surveyed position of the receiver
antenna, and it makes a more accurate estimate of the receiver clock offset by
averaging the residuals of all SVs in track.

\subsection{Usage}

\index{OrdApp!application writeup}

\subsubsection{\emph{OrdApp}}
\begin{\outputsize}
\begin{longtable}{lll}
\multicolumn{3}{c}{\application{OrdApp}} \\
\multicolumn{3}{l}{\textbf{Required Arguments}} \\
\entry{Short Arg.}{Long Arg.}{Description}{1}
\entry{-i}{--input}{Where to read the ord data. The default is stdin.}{2}
\entry{-r}{--output}{Where to write the output. The default is stdout.}{2}
\entry{-t}{--time-format}{CommonTime format specifier used for times in the output.}{3}\\
\multicolumn{3}{l}{\textbf{Optional Arguments}} \\
\entry{Short Arg.}{Long Arg.}{Description}{1}
\entry{}{--ns}{Report the clock in ns, not meters.}{1}
\end{longtable}
\end{\outputsize}

%\index{ordClock!application writeup}
\subsubsection{\emph{ordClock}}
\application{ordClock} generates clock estimates for each epoch of ORDs. \\

\begin{\outputsize}
\begin{longtable}{lll}
\multicolumn{3}{c}{\application{ordClock}} \\
\multicolumn{3}{l}{\textbf{Optional Arguments}} \\
\entry{Short Arg.}{Long Arg.}{Description}{1}
\entry{-d}{--debug}{Increase debug level.}{1}
\entry{-v}{--verbose}{Increase verbosity.}{1}
\entry{-h}{--help}{Print help usage.}{1}
\entry{-w}{--use-warts}{Use warts in the clock solution. The default is to not
                         use warts.}{2}
\entry{-e}{--estimate-only}{Only compute the receiver clock bias. Don't remove this
                         bias from the ords. The default is to both estimate the
                         bias and remove the it from the ords.}{4}
\entry{-c}{--clock-source=ARG}{An ord file to read the receiver clock offsets from.}{2}
\entry{-i}{--input=ARG}{ Where to read the ord data. The default is stdin.}{2}
\entry{-r}{--output=ARG}{Where to write the output. The default is stdout.}{2}
\entry{-t}{--time-format=ARG}{CommonTime format specifier used for times in the output.
                         The default is "\%4Y \%3j \%02H:\%02M:\%04.1f".}{3}
\entry{}{--ns}{Report the clock in ns, not meters.}{1}

\end{longtable}
\end{\outputsize}

\index{ordEdit!application writeup}

\subsubsection{\emph{ordEdit}}
\application{ordEdit} edits an ORD file based on various criteria. \\

\begin{\outputsize}
\begin{longtable}{lll}
\multicolumn{3}{c}{\application{ordEdit}} \\
\multicolumn{3}{l}{\textbf{Optional Arguments}} \\
\entry{Short Arg.}{Long Arg.}{Description}{1}
\entry{-d}{--debug}{Increase debug level.}{1}
\entry{-v}{--verbose}{Increase verbosity.}{1}
\entry{-h}{--help}{Print help usage.}{1}
\entry{-k}{--clock-est}{Remove ORDs that do not have corresponding clock estimates.}{2}
\entry{-c}{--no-clock}{Remove all clock offset estimate warts.  Give this option twice to remove all clock data.}{2}
\entry{-m}{--elev=NUM}{Remove data for SVs below a given elevation mask.}{2}
\entry{-p}{--PRN=NUM}{Filter data by PRN number. Repeat option for multiple
                        satellites. Negative PRN numbers mean exclude these
                        PRNs. Positive PRN numbers mean only include these
                        satellites. Zero removes all.}{5}
\entry{-w}{--warts=NUM}{Include/Exclude warts from the indicated PRN. Repeat
                        option for multiple PRNs. Negative numbers exclude,
                        positive numbers include, zero excludes warts from all
                        PRNs. The default is to include all warts.}{5}
\entry{-e}{--be-file=ARG}{Remove data for unhealthy SVs by providing broadcast
                        ephemeris source: RINEX nav or FIC file.}{3}
\entry{}{--start=ARG}{Throw out data before this time. Format as string: "yyyy
                        ddd HH:MM:SS".}{2}
\entry{}{--end=ARG}{Throw out data after this time. Format as string: "yyyy
                        ddd HH:MM:SS".}{2}
\entry{-s}{--size=ARG}{Remove clock residuals with absolute values greater than
                        this size (meters).}{2}
\entry{-l}{--ord-limit=ARG}{Remove ords with absolute valies greater than this size
                        (meters).}{2}
\entry{-i}{--input=ARG}{Where to read the ord data. The default is stdin.}{2}
\entry{-r}{--output=ARG}{Where to write the output. The default is stdout.}{2}
\entry{-t}{--time-format=ARG}{CommonTime format specifier used for times in the output.
                        The default is "\%4Y \%3j \%02H:\%02M:\%04.1f".}{3}
\entry{}{--ns}{Report the clock in ns, not meters.}{1}
\end{longtable}
\end{\outputsize}

\index{ordGen!application writeup}
\subsubsection{\emph{ordGen}}
\application{ordGen} generates observed range deviations.
\begin{\outputsize}
\begin{longtable}{lll}
\multicolumn{3}{c}{\application{ordGen}} \\
\multicolumn{3}{l}{\textbf{Required Arguments}} \\
\entry{Short Arg.}{Long Arg.}{Description}{1}
\entry{-o}{--obs=ARG}{Where to get the obs data.}{1}
\entry{-e}{--eph=ARG}{Where to get the ephemeris data.  Acceptable formats include RINEX (nav), FIC, MDP, SP3, YUMA, and SEM.}{2} \\
\multicolumn{3}{l}{\textbf{Optional Arguments}} \\
\entry{Short Arg.}{Long Arg.}{Description}{1}
\entry{-d}{--debug}{Increase debug level.}{1}
\entry{-v}{--verbose}{Increase verbosity.}{1}
\entry{-h}{--help}{Print help usage.}{1}
\entry{-w}{--weather=ARG}{Weather data file name (RINEX met format only).}{2}
\entry{-c}{--msc=ARG}{Station coordinate file.}{1}
\entry{}{--omode=ARG}{Specifies what observations are used to compute the
                        ORDs. Valid values are:p1p2, z1z2, c1p2, c1c2, c1y2,
                        c1z2, y1y2, c1, p1, y1, z1, c2, p2, y2, z2, smo,
                        dynamic, and smart. The default is smart.}{5}
\entry{}{--trop-model=ARG}{Specify the trop model to use. Options are zero, simple,
                        nb, and gg. The default is nb.}{2}
\entry{-p}{--pos=ARG}{Location of the antenna in meters ECEF.}{1}
\entry{-m}{--msid=NUM}{Station to process data for. Used to select a station
                        position from the msc file or data from a SMODF file.}{3}
\entry{-n}{--near}{Allows the program to select an ephemeris that is not
                        strictly in the future. Only affects the selection of
                        which broadcast ephemeris to use.}{3}
\entry{}{--sv-time}{Assume that the data is time-tagged according to each
                        SV's clock, not a common receiver clock. The is set by
                        default only for omode=smo.}{3}
\entry{-i}{--input=ARG}{Where to read the ord data. The default is stdin.}{2}
\entry{-r}{--output=ARG}{Where to write the output. The default is stdout.}{2}
\entry{-t}{--time-format=ARG}{CommonTime format specifier used for times in the output.
                        The default is "\%4Y \%3j \%02H:\%02M:\%04.1f".}{3}
\entry{}{--ns}{Report the clock in ns, not meters.}{1}
\end{longtable}
\end{\outputsize}


\index{ordLinEst!application writeup}

\subsubsection{\emph{ordLinEst}}
\application{ordLinEst} computes a linear clock estimate. \\
\begin{\outputsize}
\begin{longtable}{lll}
\multicolumn{3}{c}{\application{ordLinEst}} \\
\multicolumn{3}{l}{\textbf{Optional Arguments}} \\
\entry{Short Arg.}{Long Arg.}{Description}{1}
\entry{-d}{--debug}{Increase debug level.}{1}
\entry{-v}{--verbose}{Increase verbosity.}{1}
\entry{-h}{--help}{Print help usage.}{1}
\entry{-m}{--max-rate=ARG}{Rate used to detect a clock jump.  Default is 10,000 m/day.}{2}
\entry{-i}{--input=ARG}{Where to read the ord data. The default is stdin.}{2}
\entry{-r}{--output=ARG}{Where to write the output. The default is stdout.}{2}
\entry{-t}{--time-format=ARG}{CommonTime format specifier used for times in the output.
                        The default is "\%4Y \%3j \%02H:\%02M:\%04.1f".}{3}
\entry{}{--ns}{Report the clock in ns, not meters.}{1}
\end{longtable}
\end{\outputsize}

\index{ordStats!application writeup}

\subsubsection{\emph{ordStats}}
\application{ordStats} computes ORD statistics. \\

\begin{\outputsize}
\begin{longtable}{lll}
\multicolumn{3}{c}{\application{ordStats}} \\
\multicolumn{3}{l}{\textbf{Optional Arguments}} \\
\entry{Short Arg.}{Long Arg.}{Description}{1}
\entry{-d}{--debug}{Increase debug level.}{1}
\entry{-v}{--verbose}{Increase verbosity.}{1}
\entry{-h}{--help}{Print help usage.}{1}
\entry{-b}{--elev-bin=ARG}{A range of elevations, used in computing the statistical
                        summaries. Repeat to specify multiple bins. The default
                        is "-b 0-10 -b 10-20 -b 20-60 -b 10-90".}{4}
\entry{-s}{--sigma=NUM}{Multiplier for sigma stripping used in statistical
                        computations. The default value is 6.}{2}
\entry{-w}{--wonky}{Use wonky data in stats computation. The default is to
                        not use such data.}{2}
\entry{}{--stats-only}{Only output stats to stdout.}{1}
\entry{-i}{--input=ARG}{Where to read the ord data. The default is stdin.}{2}
\entry{-r}{--output=ARG}{Where to write the output. The default is stdout.}{2}
\entry{-t}{--time-format=ARG}{CommonTime format specifier used for times in the output.
                        The default is "\%4Y \%3j \%02H:\%02M:\%04.1f".}{3}
\entry{}{--ns}{Report the clock in ns, not meters.}{1}
\end{longtable}
\end{\outputsize}


%ToDo:
%   discuss of the need for an accurate antenna position
%   discuss of the relative errors in the broadcast versus a precise ephemeris
%   discuss the difference between SV time and RX time


\subsection{Double Difference Residuals}
While many double differences exist, reszilla computes an the first
difference to a master SV and the second difference to a second
receiver.  This double difference removes receiver clock error, iono,
trop, and SV clock errors. When the two receivers are connected to a
common antenna (often referred to as a zero-baseline setup) and are of
the same type, even the multipath is differenced out. What is left is
basically receiver tracking noise and receiver tracking errors.

One complicating factor in computing this DD is that while the clock
errors in the receivers cancel out, there is still an error associated
with the motion of the satellite during the interval between when the
two receivers computing their observation. To remove this error, an
estimate of the clock offset between the two receivers is
need. Reszilla can get this estimate in one of two ways; estimates
this by computing a clock estimate for each receiver as described
under the ORD section or reading the estimates from the rinex obs data
files. These two estimates are then differenced to get the offset
between the two receivers.

Another complicating factor is that the phase observations normally
have an "integer ambiguity" associated with them. When the DD phase
observation is computed, it will have the difference between the two
receivers ambiguity. Often this number can be quite big. Removing this
ambiguity is often referred to as debiasing the data. This process
involves much black magic and slight of hand. Do not delve into this
or even look too closely at the details or you will be sullied.

\subsection{Usage}
\index{ddGen!application writeup}
\subsubsection{\emph{ddGen}}
\application{ddGen} computes double-difference residuals from raw observations.
\begin{\outputsize}
\begin{longtable}{lll}
\multicolumn{3}{c}{\application{ddGen}} \\
\multicolumn{3}{l}{\textbf{Required Arguments}} \\
\entry{Short Arg.}{Long Arg.}{Description}{1}
\entry{-1}{--obs1=ARG}{Where to get the first receiver's obs data.}{1}
\entry{-2}{--obs2=ARG}{Where to get the second receiver's obs data.}{1}
\entry{-e}{--eph=ARG}{Where to get the ephemeris data. Acceptable formats
                           include RINEX nav, FIC, MDP, SP3, YUMA, and SEM.}{3} \\
\multicolumn{3}{l}{\textbf{Optional Arguments}} \\
\entry{Short Arg.}{Long Arg.}{Description}{1}
\entry{-d}{--debug}{Increase debug level.}{1}
\entry{-v}{--verbose}{Increase verbosity.}{1}
\entry{-h}{--help}{Print help usage.}{1}
\entry{}{--ddmode=ARG}{Specifies what observations are used to compute the
                           double difference residuals. Valid values are: all,
                           phase. The default is all.}{3}
\entry{}{--omode=ARG}{Specifies what observations to use to compute the
                           ORDs. Valid values are: p1p2, z1z2, c1p2, c1y2, c1z2,
                           y1y2, c1, p1, y1, z1, c2, p2, y2, z2 smo, dynamic,
                           and smart. The default is smart.}{5}
\entry{}{--min-arc-time=ARG}{The minimum length of time (in seconds) that a
                           sequence of observations must span to be considered
                           as an arc. The default value is 60.0 seconds.}{3}
\entry{}{--min-arc-gap=ARG}{The minimum length of time (in seconds) between two
                           arcs for them to be considered separate arcs. The
                           default value is 60.0 seconds.}{4}
\entry{}{--min-arc-length=ARG}{The minimum number of epochs that can be considered
                           an arc. The default value is 5 epochs.}{2}
\entry{}{--noise=ARG}{The noise threshold used in finding discontinuitites.
                           The default is 0.1000 cycles.}{2}
\entry{-b}{--elev-bin=ARG}{Range of elevations to use in computing the
                           statistical summaries. Repeat to specify multiple
                           bins. The default is "-b 0-10 -b 10-20 -b 20-60 -b
                           10-90".}{4}
\entry{-c}{--msc=ARG}{Station coordinate file.}{1}
\entry{-p}{--pos=ARG}{Location of the antenna in meters ECEF.}{1}
\entry{-E}{--health-src=ARG}{Do not use data from unhealthy SVs as determined
                           using this ephemeris source. Can be RINEX navigation
                           or FIC file(s).}{3}
\entry{}{--strip=ARG}{Factor used in stripping data prior to computing
                           descriptive statistics. The default value is 3.2.}{2}
\entry{}{--phase=ARG}{Only compute phase double differences.}{1}
\entry{}{-S}{--SNR=ARG}{Only included observables with a raw signal strength,
                           or SNR, of at least this value, in dB. The default is
                           20 dB.}{3}
\entry{-m}{--msid=NUM}{Station to process data for. Used to select a station
                           position from the msc file or data from a SMODF file.}{3}
\entry{-w}{--window=NUM}{Compute mean values of the double differences over
                           this time span (seconds). (15 min = 900)}{2}
\entry{-r}{--raw}{Output the raw double differences in addition to the
                           descriptive statistics.}{2}
\entry{-a}{--all-combos}{Compute all combinations, don't just use one master
                           SV.}{2}
\entry{-n}{--near}{Allow the program to select an ephemeris that is not
                           strictly in the future. Only affects the selection of
                           which broadcast ephemeris to use. i.e. use a close
                           ephemeris.}{5}
\entry{}{--zero-trop}{Disables trop corrections.}{1}
\end{longtable}
\end{\outputsize}

\subsection{Data Input}

Several different types of data are required to compute these
residuals; the raw observations, the receiver antenna position, the
satellite position, and optionally  weather observations. The raw
observations may be supplied to reszilla in one of several formats;
rinex obs (see RinexObsData class), smodf (see SMODFData class), and
MDP (see MDPObsEpoch class in apps/MDPtools). The reciever antenna
postion may be specified in the rinex obs header or via a station
coordinates file (see MSCData class).

\subsection{Output}
There are two general types of output that reszilla produces -
statistical summaries and the raw residuals.  The mean, standard 
deviation, and maximum value of the residuals are calculated 
as a function of specified elevation ranges and are output in a 
statistics table. Looking at the results for each elevation bin 
is useful as ORDs tend to be much a higher when satellites are 
lower on the horizon. For a more thorough analysis, the ORD or DD 
residuals calculated by reszilla may be output in a matrix format 
to a file with columns for time, PRN, elevation, ORD or clock residual, 
IODC, satellite health, and a flag for the residual type.  The flag 
specifies exactly which of the 13 possible residual types the data 
on that row represent, depending on the method used for calculation. 

One benefit of this output feature is that residuals can be looked at 
for particular time periods or PRNs. Fortunately there is a companion 
plotting tool that makes this simple. Given a reszilla output file, 
the dplot program will plot residuals and, if specified, receiver clock 
estimates versus time using gnuplot. A user may specify the time 
range, stripping value, and PRN(s) to use in the plot, as well as a
filename for saving the result. 

%Examples:

 %  < to be added later >
 %  smodf with met and sp3
 %  rinex-30 with met and sp3
 %  rinex-30 without met with sp3
 %  rinex-30 with met and fic
 %  rinex-30 without met with fic

\begin{\outputsize}
\begin{verbatim}
Types in the raw output files:
   0 - c1p2 observed range deviation
   50 - computed clock, difference from estimate, strip
   51 - linear clock estimate, abdev
Double difference types:
   10 - c1     20 - c2
   11 - p1     21 - p2
   12 - l1     22 - l2
   13 - d1     23 - d2
   14 - s1     24 - s2
\end{verbatim}
\end{\outputsize}

\subsection{Notes}
The criteria min-arc-time and min-arc-length are both required to be met
for a arc to be valid in double difference mode.
All output quantities (stddev, min, max, ord, clock, double differnce, ...)
are in meters.

%\end{document}

