%\documentclass{article}
%\usepackage{fancyvrb}
%\usepackage{src/perltex}
%\usepackage{src/xcolor}
%\usepackage{listings}
%\usepackage{longtable}
%\usepackage{multirow}
%\RecustomVerbatimEnvironment{Verbatim}{Verbatim}{frame=single}

\newcommand{\outputsize}{footnotesize}

\perlnewcommand{\getuse}[1]
{
        my $command = $_[0];
        $command = $command." > temp 2>&1";
        system("$command");

	my $output = "";
        open(input,"temp");
        while(my $line = <input>){$output = $output.$line;}
        close(input);

        return  "\\begin{\\outputsize}\n" . "\\begin{Verbatim}\n" .
                $output .
                "\\end{Verbatim}\n" . "\\end{\\outputsize}\n";
}

\perlnewcommand{\printexec}[1]
{
	my $exec = $_[0];

	return "\\begin{\\outputsize}\n" . "\\begin{Verbatim}\n" .
		$exec . "\n" .
		"\\end{Verbatim}\n" . "\\end{\\outputsize}\n";
}

\perlnewcommand{\application}[1]
{
	my $app = $_[0];
	return "\\emph{" . $app . "}";
}

%\begin{document}

\index{rtAshtech!application writeup}
\section{\emph{rtAshtech}}
\subsection{Overview}
This application logs observations from an Ashtech Z-XII receiver. It records
observations directly into the RINEX format. A number of optional outputs are
possible. The raw messages from a receiver can be recorded. Observations can
also be recorded in a format that is easily imported into numerical packages.

\subsection{Usage}
\begin{\outputsize}
\begin{longtable}{lll}
\multicolumn{3}{l}{\application{rtAshtech}} \\
\multicolumn{3}{l}{\textbf{Optional Arguments}} \\
\entry{Short Arg.}{Long Arg.}{Description}{1}
\entry{-h}{--help}{Print help usage}{1}
\entry{-v}{--verbose}{Increased diagnostic messages}{1}
\entry{-r}{--raw}{Record raw observations}{1}
\entry{-l}{--log}{Record log entries}{1}
\entry{-t}{--text}{Record observations as simple text files}{1}
\entry{-p}{--port=ARG}{Serial port to use}{1}
\entry{-o}{--rinex-obs=ARG}{Naming convention for RINEX obs files}{1}
\entry{-n}{--rinex-nav=ARG}{Naming convention for RINEX nav message files}{1}
\entry{-T}{--text-obs=ARG}{ Naming convention for obs in simple text files}{1}
\end{longtable}
\end{\outputsize}

\subsection{Examples}
\begin{\outputsize}
\begin{lstlisting}
> rtAshtech -p /dev/ttyS1
\end{lstlisting}
\end{\outputsize}

\begin{\outputsize}
\begin{lstlisting}
> rtAshtech -o "minute\%03j\%02H\%02M.\%02yo"
\end{lstlisting}
\end{\outputsize}

\subsection{Notes}
Only works on UNIX systems with POSIX compliant serial ports.

%\end{document}

