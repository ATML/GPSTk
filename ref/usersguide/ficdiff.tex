%\documentclass{article}
%\usepackage{fancyvrb}
%\usepackage{src/perltex}
%\usepackage{src/xcolor}
%\usepackage{listings}
%\usepackage{longtable}
%\usepackage{multirow}
%\RecustomVerbatimEnvironment{Verbatim}{Verbatim}{frame=single}

\newcommand{\outputsize}{footnotesize}

\perlnewcommand{\getuse}[1]
{
        my $command = $_[0];
        $command = $command." > temp 2>&1";
        system("$command");

	my $output = "";
        open(input,"temp");
        while(my $line = <input>){$output = $output.$line;}
        close(input);

        return  "\\begin{\\outputsize}\n" . "\\begin{Verbatim}\n" .
                $output .
                "\\end{Verbatim}\n" . "\\end{\\outputsize}\n";
}

\perlnewcommand{\printexec}[1]
{
	my $exec = $_[0];

	return "\\begin{\\outputsize}\n" . "\\begin{Verbatim}\n" .
		$exec . "\n" .
		"\\end{Verbatim}\n" . "\\end{\\outputsize}\n";
}

\perlnewcommand{\application}[1]
{
	my $app = $_[0];
	return "\\emph{" . $app . "}";
}

%\begin{document}

\index{ficdiff!application writeup}
\section{\emph{ficdiff}}
\subsection{Overview}
The application compares the contents of two FIC files containing ephemeris data.

\subsection{Usage}
\begin{\outputsize}

\begin{longtable}{lll}
\multicolumn{3}{l}{\textbf{Optional Arguments}} \\
\entry{Short Arg.}{Long Arg.}{Description}{1}
\entry{-d}{--debug}{Increase debug level}{1}
\entry{-v}{--verbose}{Increase verbosity}{1}
\entry{-h}{--help}{Print help usage}{1}
\entry{-t}{--time=TIME}{Start of time range to compare (default = "beginning of time")}{2}
\entry{-e}{--end-time=TIME}{End of time range to compare (default = "end of time")}{2}
& & \\
\multicolumn{3}{c}{ephdiff usage: ficdiff [options] fic1 fic2} \\
\end{longtable}
\end{\outputsize}

\subsection{Examples}
\begin{\outputsize}
\begin{lstlisting}
> ficdiff -t "08/01/2006 12:00:00" fic1 fic2
<FIC BlockNumber: 9
 floats:   139 362 172806 1 1 1386 1 0 0 55296 0 -4.19095e-09 180000 0 . . .
 integers:
 chars:

<FIC BlockNumber: 9
 floats:   139 362 172806 1 1 1386 1 0 0 59392 0 -6.98492e-09 179984 0 . . .
 integers:
 chars:
 . . .

\end{lstlisting}
\end{\outputsize}

\subsection{Notes}

%\end{document}
