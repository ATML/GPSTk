%\documentclass{article}
%\usepackage{fancyvrb}
%\usepackage{perltex}
%\usepackage{xcolor}
%\usepackage{listings}
%\usepackage{longtable}
%\usepackage{multirow}
%\RecustomVerbatimEnvironment{Verbatim}{Verbatim}{frame=single}

\newcommand{\outputsize}{footnotesize}

\perlnewcommand{\getuse}[1]
{
        my $command = $_[0];
        $command = $command." > temp 2>&1";
        system("$command");

	my $output = "";
        open(input,"temp");
        while(my $line = <input>){$output = $output.$line;}
        close(input);

        return  "\\begin{\\outputsize}\n" . "\\begin{Verbatim}\n" .
                $output .
                "\\end{Verbatim}\n" . "\\end{\\outputsize}\n";
}

\perlnewcommand{\printexec}[1]
{
	my $exec = $_[0];

	return "\\begin{\\outputsize}\n" . "\\begin{Verbatim}\n" .
		$exec . "\n" .
		"\\end{Verbatim}\n" . "\\end{\\outputsize}\n";
}

\perlnewcommand{\application}[1]
{
	my $app = $_[0];
	return "\\emph{" . $app . "}";
}

%\begin{document}

\index{poscvt!application writeup}
\section{\emph{poscvt}}
\subsection{Overview}
This application allows the user to convert among different coordinate system on 
the command line. Coordinate systems handled include Cartesian, geocentric, and 
geodetic.

\subsection{Usage}
\begin{\outputsize}
\begin{longtable}{lll}
\multicolumn{3}{c}{\application{poscvt}} \\
\multicolumn{3}{l}{\textbf{Optional Arguments}} \\
\entry{Short Arg.}{Long Arg.}{Description}{1}
\entry{-d}{--debug}{Increase debug level.}{1}
\entry{-v}{--verbose}{Increase verbosity.}{1}
\entry{-h}{--help}{Print help usage.}{1}
\entry{}{--ecef=POSITION}{ECEF ``X Y Z'' in meters.}{1}
\entry{}{--geodetic=POSITION}{Geodetic ``lat lon alt'' in deg, deg, meters.}{1}
\entry{}{--geocentric=POSITION}{Geocentric ``lat lon radius'' in deg, deg, meters.}{1}
\entry{}{--spherical=POSITION}{Spherical ``theta, pi, radius'' in deg, deg, meters.}{1}
\entry{-l}{--list-formats}{List the available format codes for use by the input and output format options.}{2}
\entry{-F}{--output-format=ARG}{Write the position with the given format.}{1}
\end{longtable}
\end{\outputsize}

\subsection{Examples}
\begin{\outputsize}
\begin{verbatim}
> poscvt --ecef="4345070.59253 45619878.26297 803.598856837"

    ECEF (x,y,z) in meters              4345070.5925 45619878.2630 803.5989
    Geodetic (llh) in deg, deg, m       0.00100566 84.55926933 39448197.4795
    Geocentric (llr) in deg, deg, m     0.00100472 84.55926933 45826334.4795
    Spherical (tpr) in deg, deg, m      89.99899528 84.55926933 45826334.4795

\end{verbatim}
\end{\outputsize}
\subsection{Notes}
If no options are given \application{poscvt} assumes XYZ 0 0 0.

%\end{document}

