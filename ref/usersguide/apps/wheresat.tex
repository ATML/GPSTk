\section{\emph{wheresat}}
\subsection{Usage}
\getuse{wheresat --help}
\subsection{Examples}
\begin{verbatim}

> wheresat -b aira1720.06n -p 2 -u "918129.01 -4346070.45 803.18" 
  -s "06/21/2006 17:00:00" -e "06/21/2006 20:00:00" -t 1800 

 Antenna Position:  918129  -4.34607e+06  803.18
 Navigation File:   aira1720.06n
 Start Time:        06/21/2006 17:00:00
 End Time:          06/21/2006 20:00:00
 PRN:               2

 Prn 2 Earth-fixed position and clock information:

 Date       Time(UTC)   X (meters)          Y (meters)
 =========================================================
 06/21/2006 18:00:00  12758891.971859      18901201.616227
 06/21/2006 18:30:00  12847888.097031      21541501.416411
 06/21/2006 19:00:00  12843576.989405      23087218.618683
 06/21/2006 19:30:00  12450313.769289      23516935.034029

  . . .

 Z (meters)          Clock Correc (s)
 ==================================
 -14049016.596144     0.000007
 -9315422.851798      0.000007
 -3957280.515764      0.000007
 1667186.089065       0.000007



 Data for user reference frame:

 Date       Time(UTC)   Azimuth        Elevation      Range to SV (m)
 =====================================================================
 06/21/2006 18:00:00  130.596202      -43.242769      29627531.177821
 06/21/2006 18:30:00  118.680085      -49.681012      29983796.522429
 06/21/2006 19:00:00  102.845663      -53.888528      30169796.433699
 06/21/2006 19:30:00  84.400419       -55.459042      30197072.648367

 Calculated 4 increments for prn 2 .

\end{verbatim}
\subsection{Notes}
Wheresat uses the broadcast ephemeris within the given Rinex navigation file to compute the satellite positions.
