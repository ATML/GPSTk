%\documentclass{article}
%\usepackage{fancyvrb}
%\usepackage{perltex}
%\usepackage{xcolor}
%\usepackage{listings}
%\usepackage{longtable}
%\usepackage{multirow}
%\RecustomVerbatimEnvironment{Verbatim}{Verbatim}{frame=single}

\newcommand{\outputsize}{footnotesize}

\perlnewcommand{\getuse}[1]
{
        my $command = $_[0];
        $command = $command." > temp 2>&1";
        system("$command");

	my $output = "";
        open(input,"temp");
        while(my $line = <input>){$output = $output.$line;}
        close(input);

        return  "\\begin{\\outputsize}\n" . "\\begin{Verbatim}\n" .
                $output .
                "\\end{Verbatim}\n" . "\\end{\\outputsize}\n";
}

\perlnewcommand{\printexec}[1]
{
	my $exec = $_[0];

	return "\\begin{\\outputsize}\n" . "\\begin{Verbatim}\n" .
		$exec . "\n" .
		"\\end{Verbatim}\n" . "\\end{\\outputsize}\n";
}

\perlnewcommand{\application}[1]
{
	my $app = $_[0];
	return "\\emph{" . $app . "}";
}

%\begin{document}

\index{WhereSat!application writeup}
\section{\emph{WhereSat}}
\subsection{Overview}
This application uses input ephemeris to compute the predicted location of a 
satellite. The Earth-centered, Earth-fixed (ECEF) position of the satellite is 
reported. Optionally, the topocentric coordinates--azimuth, elevation, and 
range--can be generated. The user can specify the time interval between 
successive predictions. Also the output can generated in a format easily
imported into numerical packages.

\subsection{Usage}
\begin{\outputsize}
\begin{longtable}{lll}
\multicolumn{3}{c}{\application{WhereSat}} \\
\multicolumn{3}{l}{\textbf{Required Arguments}} \\
\entry{Short Arg.}{Long Arg.}{Description}{1}
\entry{-e}{--eph-files=ARG}{Ephemeris source file(s).  Can be RINEX nav, SP3, or FIC.}{1}
& & \\
\multicolumn{3}{l}{\textbf{Optional Arguments}} \\
\entry{Short Arg.}{Long Arg.}{Description}{1}
\entry{-h}{--help}{Print help usage.}{1}
\entry{-u}{--position=ARG}{Antenna position in ECEF (x,y,z) coordinates.  Format as string: ``X Y Z".  used to give user-centered data (SV range, azimuth, and elevation) when SV is in view.}{4}
\entry{}{--start=ARG}{Ignore data before this time.  Format as string: ``MO/DD/YYYY HH:MM:SS".}{2}
\entry{}{--end=ARG}{Ignore data after this time.  Format as string: ``MO/DD/YYYY HH:MM:SS".}{2}
\entry{-f}{--time-format=ARG}{CommonTime format specifier used for times in the output.  The default is ``\%4Y \%3j \%02H:\%02M:\%4.1f".}{3}
\entry{-p}{--prn=NUM}{Which SVs to analyze.  Repeat option for multiple satellites.  If this option is not specified, all ephemeris data will be processed.}{3}
\entry{-t}{--time=NUM}{Time increment in seconds for ephemeris calculation.  Default is 900 seconds (15 minutes).}{2}
\end{longtable}
\end{\outputsize}

\subsection{Examples}
\begin{\outputsize}
\begin{lstlisting}
> WhereSat -b aira1720.06n -p 2 -u "918129.01 -4346070.45 803.18"
  -s "06/21/2006 17:00:00" -e "06/21/2006 20:00:00" -t 1800

 Antenna Position:  918129  -4.34607e+06  803.18
 Navigation File:   aira1720.06n
 Start Time:        06/21/2006 17:00:00
 End Time:          06/21/2006 20:00:00
 PRN:               2

 Prn 2 Earth-fixed position and clock information:

 Date       Time(UTC)   X (meters)          Y (meters)          Z (meters)      
 ===============================================================================
 06/21/2006 18:00:00  12758891.971859      18901201.616227      -14049016.596144
 06/21/2006 18:30:00  12847888.097031      21541501.416411      -9315422.851798 
 06/21/2006 19:00:00  12843576.989405      23087218.618683      -3957280.515764 
 06/21/2006 19:30:00  12450313.769289      23516935.034029      1667186.089065  

  . . .

    Clock Correc (s)
==================
     0.000007
     0.000007
     0.000007
     0.000007

 

 Data for user reference frame:

 Date       Time(UTC)   Azimuth        Elevation      Range to SV (m)
 =====================================================================
 06/21/2006 18:00:00  130.596202      -43.242769      29627531.177821
 06/21/2006 18:30:00  118.680085      -49.681012      29983796.522429
 06/21/2006 19:00:00  102.845663      -53.888528      30169796.433699
 06/21/2006 19:30:00  84.400419       -55.459042      30197072.648367

 Calculated 4 increments for prn 2 .


\end{lstlisting}
\end{\outputsize}


%\end{document}

