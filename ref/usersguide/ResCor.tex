%\documentclass{article}
%\usepackage{fancyvrb}
%\usepackage{perltex}
%\usepackage{xcolor}
%\usepackage{listings}
%\usepackage{multirow}
%\usepackage{longtable}
%\RecustomVerbatimEnvironment{Verbatim}{Verbatim}{frame=single}

\newcommand{\outputsize}{footnotesize}

\perlnewcommand{\getuse}[1]
{
        my $command = $_[0];
        $command = $command." > temp 2>&1";
        system("$command");

	my $output = "";
        open(input,"temp");
        while(my $line = <input>){$output = $output.$line;}
        close(input);

        return  "\\begin{\\outputsize}\n" . "\\begin{Verbatim}\n" .
                $output .
                "\\end{Verbatim}\n" . "\\end{\\outputsize}\n";
}

\perlnewcommand{\printexec}[1]
{
	my $exec = $_[0];

	return "\\begin{\\outputsize}\n" . "\\begin{Verbatim}\n" .
		$exec . "\n" .
		"\\end{Verbatim}\n" . "\\end{\\outputsize}\n";
}

\perlnewcommand{\application}[1]
{
	my $app = $_[0];
	return "\\emph{" . $app . "}";
}

%\begin{document}

\index{ResCor!application writeup}
\section{\emph{ResCor}}
\subsection{Overview}
The application will open and read a single RINEX observation file, apply editing commands
   using the RinexEditor package, compute any of several residuals and corrections and
   register extended RINEX observation types for them, and then write the edited data,
   along with the new extended observation types, to an output RINEX observation file.
\\
\\
\textbf{NOTE: ResCor is only available in GPSTK 1.x. It is only compatiable with Rinex versions 2.1 and earlier}

\subsection{Usage}
\subsubsection{\emph{ResCor}}

\begin{\outputsize}

\begin{longtable}{lll}
%\multicolumn{3}{c}{\application{ResCor}} \\
\multicolumn{3}{l}{\textbf{Required Arguments}} \\
\entry{Short Arg.}{Long Arg.}{Description}{1}
\entry{-IF}{}{Input RINEX observation file.}{1}
\entry{-OF}{}{Name of ouput RINEX observation file.}{1}
& & \\
\multicolumn{3}{l}{\textbf{Configuration Arguments}} \\
\entry{Short Arg.}{Long Arg.}{Description}{1}
\entry{-f$<$file$>$}{}{File containing more options.}{1}
\entry{}{--nav $<$file$>$}{Navigation (RINEX Nav OR SP3) file(s).}{1}
\entry{}{--navdir $<$dir$>$}{Directory of navigation file(s).}{1}
& & \\
\multicolumn{3}{l}{\textbf{Reference Position Input}} \\
\entry{Short Arg.}{Long Arg.}{Description}{1}
\entry{}{--RxLLH $<$l,l,h$>$}{1.Receiver position (static) in geodetic lat, lon(E), ht (deg,deg,m).}{2}
\entry{}{--RxXYZ $<$x,y,z$>$}{2.Receiver position (static) in ECEF coordinates (m).}{2}
\entry{}{--Rxhere}{3.Reference site positions(time) from this file (i.e. -IF$<$RINEXFile$>$).}{2}
\entry{}{--RxRinex $<$fn$>$}{4.Reference site positions(time) from another RINEX file named $<$fn$>$.}{2}
\entry{}{--RxFlat $<$fn$>$}{5.Reference site positions and times given in a flat file named $<$fn$>$.}{2}
\entry{}{--Rxhelp}{(Enter --Rxhelp for a description of the -RxFlat file format).}{2}
\entry{}{--RAIM}{6.Reference site positions computed via RAIM (requires P1,P2,EP).}{2}
\entry{}{}{NB the following two options apply only if --RAIM is found.}{2}
\entry{}{--noRAIMedit}{Do not edit data based on RAIM solution.}{1}
\entry{}{--RAIMhead}{Output average RAIM solution to RINEX header (if -HDf also appears).}{2}
\entry{}{--noRefout}{Do not output reference solution to RINEX.}{1}
\entry{}{--MinElev}{Minimum satellite elevation in degrees for output.}{2}
& & \\
\multicolumn{3}{l}{\textbf{Residual/Correction Computation}} \\
\entry{Short Arg.}{Long Arg.}{Description}{1}
\entry{}{--debias $<$OT,l$>$}{Debias new output type $<$OT$>$; trigger a bias reset with limit $<$l$>$.}{2}
\entry{}{--Callow}{Allow C1 to replace P1 when P1 is not available.}{1}
\entry{}{--Cforce}{Force C/A code pseudorange C1 to replace P1.}{1}
\entry{}{--IonoHt $<$ht$>$}{Height of ionosphere in km (default 400) (needed for LA,LO,VR,VP).}{2}
\entry{}{--Tgd}{Apply the Tgd from BC ephemeris to SR,SP,VR, and VP.}{2}
\entry{}{--SVonly $<$prn$>$}{Process this satellite ONLY.}{1}
& & \\
\multicolumn{3}{l}{\textbf{Output Files}} \\
\entry{Short Arg.}{Long Arg.}{Description}{1}
\entry{}{--Log $<$file$>$}{Output log file name (rc.log)}{1}
& & \\
\multicolumn{3}{l}{\textbf{Help}} \\
\entry{Short Arg.}{Long Arg.}{Description}{1}
\entry{}{--verbose}{Print extended output}{1}
\entry{}{--debug}{Print debugging information.}{1}
\entry{-h}{--help}{Print syntax and quit.}{1}
\entry{}{--REChelp}{Print syntax of RINEXEditor commands and quit.}{2}
\entry{}{--ROThelp}{Print list of extended RINEX observation types and quit.}{2}
\end{longtable}
\end{\outputsize}

\subsubsection{List of Available RINEX Observation Types}
\begin{\outputsize}
\begin{verbatim}
  OT Description          Units     Required input (EP=ephemeris,PS=Rx Position)
  ER Ephemeris range      meters                 EP PS
  RI Iono Delay, Range    meters              P1
  PI Iono Delay, Phase    meters     L1 L2
  TR Tropospheric Delay   meters                 EP PS
  RL Relativity Correct.  meters                 EP
  SC SV Clock Bias        meters                 EP
  EL Elevation Angle      degrees                EP PS
  AZ Azimuth Angle        degrees                EP PS
  SR Slant TEC (PR)       TECU                P1
  SP Slant TEC (Ph)       TECU       L1 L2
  VR Vertical TEC (PR)    TECU                P1 EP PS
  VP Vertical TEC (Ph)    TECU       L1 L2       EP PS
  LA Lat Iono Intercept   degrees                EP PS
  LO Lon Iono Intercept   degrees                EP PS
  P3 TFC(IF) Pseudorange  meters              P1
  L3 TFC(IF) Phase        meters     L1 L2
  P4 GeoFree Pseudorange  meters              P1
  L4 GeoFree Phase        meters     L1 L2
  P5 WideLane Pseudorange meters              P1
  L5 WideLane Phase       meters     L1 L2
  MP Multipath (=M3)      meters     L1 L2    P1
  M1 L1 Range minus Phase meters     L1       P1
  M2 L2 Range minus Phase meters        L2
  M3 IF Range minus Phase meters     L1 L2    P1
  M4 GF Range minus Phase meters     L1 L2    P1
  M5 WL Range minus Phase meters     L1 L2    P1
  XR Non-dispersive Range meters     L1 L2    P1
  XI Ionospheric delay    meters     L1 L2    P1
  X1 Range Error L1       meters     L1 L2    P1
  X2 Range Error L2       meters     L1 L2    P1
  SX Satellite ECEF-X     meters                 EP
  SY Satellite ECEF-Y     meters                 EP
  SZ Satellite ECEF-Z     meters                 EP
\end{verbatim}
\end{\outputsize}

%\end{document}
