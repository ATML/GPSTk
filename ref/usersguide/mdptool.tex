%\documentclass{article}
%\usepackage{fancyvrb}
%\usepackage{src/perltex}
%\usepackage{src/xcolor}
%\usepackage{listings}
%\usepackage{longtable}
%\usepackage{multirow}
%\RecustomVerbatimEnvironment{Verbatim}{Verbatim}{frame=single}

\newcommand{\outputsize}{footnotesize}

\perlnewcommand{\getuse}[1]
{
        my $command = $_[0];
        $command = $command." > temp 2>&1";
        system("$command");

	my $output = "";
        open(input,"temp");
        while(my $line = <input>){$output = $output.$line;}
        close(input);

        return  "\\begin{\\outputsize}\n" . "\\begin{Verbatim}\n" .
                $output .
                "\\end{Verbatim}\n" . "\\end{\\outputsize}\n";
}

\perlnewcommand{\printexec}[1]
{
	my $exec = $_[0];

	return "\\begin{\\outputsize}\n" . "\\begin{Verbatim}\n" .
		$exec . "\n" .
		"\\end{Verbatim}\n" . "\\end{\\outputsize}\n";
}

\perlnewcommand{\application}[1]
{
	my $app = $_[0];
	return "\\emph{" . $app . "}";
}

%\begin{document}

\index{mdptool!application writeup}

\section{\emph{mdptool}}
\subsection{Overview}
The application performs verious functions on a stream of MDP data.

\subsection{Usage}
\begin{\outputsize}
\begin{longtable}{lll}
\multicolumn{3}{c}{\application{mdptool}} \\
\multicolumn{3}{l}{\textbf{Optional Arguments}} \\
\entry{Short Arg.}{Long Arg.}{Description}{1}
\entry{-d}{--debug}{Increase debug level}{1}
\entry{-v}{--verbose}{Increase verbosity}{1}
\entry{-h}{--help}{Print help usage}{1}
\entry{-i}{--input=ARG}{Where to get the MDP data from. The default is to use
                         stdin. If the file name begins with "tcp:" the
                         remainder is assumed to be a hostname[:port] and the
                         source is taken from a tcp socket at this address. If
                         the port number is not specified a default of 8910 is
                         used.}{6}
\entry{}{--output=ARG}{Where to send the output. The default is stdout.}{2}
\entry{-p}{--pvt}{Enable pvt output}{1}
\entry{-o}{--obs}{Enable obs output}{1}
\entry{-n}{--nav}{Enable nav output}{1}
\entry{-t}{--test}{Enable selftest output}{1}
\entry{-x}{--hex}{Dump all messages in hex}{1}
\entry{-b}{--bad}{Try to process bad messages also.}{1}
\entry{-a}{--almanac}{Build and process almanacs. Only applies to the nav style}{2}
\entry{-e}{--ephemeris}{Build and process engineering ephemerides. Only applies to the nav style}{2}
\entry{-s}{--output-style=ARG}{What type of output to produce from the MDP stream.
                         Valid styles are: brief, verbose, table, track, null,
                         mdp, nav, and summary. The default is summary. Some
                         modes aren't quite complete. Sorry.}{5}
\entry{-l}{--timeSpan=NUM}{How much data to process, in seconds}{1}
\entry{}{--startTime=TIME}{Ignore data before this time. (\%4Y/\%03j/\%02H:\%02M:\%05.2f)}{2}
\entry{}{--stopTime=TIME}{Ignore any data after this time}{1}

\end{longtable}
\end{\outputsize}

\subsection{Examples}

\subsection{Notes}
In the summary mode, the
default is to only summarize the obs data above 10 degrees. Increasing the
verbosity level will also summarize the data below 10 degrees.

%\end{document}
