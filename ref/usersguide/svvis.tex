%\documentclass{article}
%\usepackage{fancyvrb}
%\usepackage{perltex}
%\usepackage{xcolor}
%\usepackage{listings}
%\usepackage{multirow}
%\usepackage{longtable}
%\RecustomVerbatimEnvironment{Verbatim}{Verbatim}{frame=single}

\newcommand{\outputsize}{footnotesize}

\perlnewcommand{\getuse}[1]
{
        my $command = $_[0];
        $command = $command." > temp 2>&1";
        system("$command");

	my $output = "";
        open(input,"temp");
        while(my $line = <input>){$output = $output.$line;}
        close(input);

        return  "\\begin{\\outputsize}\n" . "\\begin{Verbatim}\n" .
                $output .
                "\\end{Verbatim}\n" . "\\end{\\outputsize}\n";
}

\perlnewcommand{\printexec}[1]
{
	my $exec = $_[0];

	return "\\begin{\\outputsize}\n" . "\\begin{Verbatim}\n" .
		$exec . "\n" .
		"\\end{Verbatim}\n" . "\\end{\\outputsize}\n";
}

\perlnewcommand{\application}[1]
{
	my $app = $_[0];
	return "\\emph{" . $app . "}";
}

%\begin{document}

\index{svvis!application writeup}
\section{\emph{svvis}}
\subsection{Overview}
This application computes when satellites are visible at a given point on the earth.

\subsection{Usage}
\subsubsection{\emph{svvis}}
\begin{\outputsize}
\begin{longtable}{lll}
%\multicolumn{3}{c}{\application{svvis}} \\
\multicolumn{3}{l}{\textbf{Required Arguments}} \\
\entry{Short Arg.}{Long Arg.}{Description}{1}
\entry{-e}{--eph=ARG}{Where to get the ephemeris data.  Can be RINEX, nav, FIC, MDP, SP3, YUMA, and SEM.}{3}
& & \\
\multicolumn{3}{l}{\textbf{Optional Arguments}} \\
\entry{Short Arg.}{Long Arg.}{Description}{1}
\entry{-d}{--debug}{Increase debug level.}{1}
\entry{-v}{--verbose}{Increase verbosity.}{1}
\entry{-h}{--help}{Print help usage.}{1}
\entry{}{--elevation-mask=ARG}{The elevation above which an SV is visible.  The default is 0 degrees.}{2}
\entry{-p}{--position=ARG}{Receiver antenna position in ECEF (x,y,z) coordinates.  Format as string: ``X Y Z".}{2}
\entry{-c}{--msc=ARG}{Station coordinate file.}{1}
\entry{-m}{--msid=ARG}{Station number to use from the msc file.}{1}
\entry{}{--graph-elev=ARG}{Output data at the specified interval.  Interval is in seconds.}{2}
\entry{-l}{--time-span=ARG}{How much data to process, in seconds.  Default is 86400.}{2}
\entry{}{--start-time=TIME}{When to start computing positions.  The default is the start of the ephemeris data.}{2}
\entry{}{--stop-time=TIME}{When to stop computing positions.  The default is one day after the start time.}{2}
\entry{}{--print-elev}{Print the elevation of the sv at each change in tracking.  The default is just to ouput the PRN of the SV.}{3}
\entry{}{--rise-set}{Print the visibility data by PRN in rise-set pairs.}{1}
\entry{}{--tabular}{Print the visibility data in a tabular format.}{1}
\entry{}{--recent-eph}{Use this if the ephemeris data provided uses 10-bit GPS weeks and it should be converted to the current epoch or to the epoch current to the ``start-time", if specified.}{4}
\end{longtable}
\end{\outputsize}

\subsection{Examples}
\begin{verbatim}
> svvis -e nav/s121001a.09n -p -3939182.6018,3467075.4175,-3613220.2782,402 --tabular

SEARCH_INTERVAL: 2009 001 00:00:00 to 2009 002 02:00:00
ELEVATION_CUTOFF: 0.000
#     Rise (Yr DOY HMS) Set  (Yr DOY HMS) El Sys          Parameters
PASS: 2009 001 00:12:55 2009 001 03:59:59 80 GPS          PRN=23
PASS: 2009 001 01:23:30 2009 001 05:59:43 62 GPS          PRN=25
PASS: 2009 001 01:37:18 2009 001 05:59:59 77 GPS          PRN=13
PASS: 2009 001 02:33:15 2009 001 03:59:59 16 GPS          PRN=28
PASS: 2009 001 02:41:00 2009 001 05:59:59 53 GPS          PRN=07
PASS: 2009 001 03:48:56 2009 001 05:59:59 43 GPS          PRN=17
PASS: 2009 001 04:14:43 2009 001 07:59:59 77 GPS          PRN=08
PASS: 2009 001 04:41:56 2009 001 07:59:59 78 GPS          PRN=27
PASS: 2009 001 05:26:57 2009 001 07:59:59 56 GPS          PRN=04
PASS: 2009 001 07:00:06 2009 001 11:59:59 25 GPS          PRN=26
PASS: 2009 001 07:16:58 2009 001 07:59:59 15 GPS          PRN=02
...
\end{verbatim}

%\end{document}
