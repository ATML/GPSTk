%\begin{titlepage}
The goal of the GPSTk project is to provide a world class, open source computing suite to the satellite navigation community. It is our hope that the GPSTk will empower its users to perform new research and to create new applications.

GPS users employ practically every computational architecture and operating system. Therefore the design of the GPSTk suite is as platform-independent as possible. Platform independence is achieved through use of the ANSI-standard C++ programming language. The principles of object-oriented programming are used throughout the GPSTk code base in order to ensure that the code is modular, extensible, and maintainable.

The GPSTk suite consists of a core library, auxiliary libraries, and a set of applications. The core library provides a wide array of functions that solve processing problems associated with GPS such as processing or using RINEX. The library is the basis for the more advanced applications distributed as part of the GPSTk suite.

The GPSTk is sponsored by Space and Geophysics Laboratory, within the Applied Research Laboratories at the University of Texas at Austin (ARL:UT). GPSTk is the by-product of GPS research conducted at ARL:UT since before the first satellite launched in 1978; it is the combined effort of many software engineers and scientists. In 2003 the research staff at ARL:UT decided to open source much of their basic GPS processing software as the GPSTk.
%\end{titlepage}
