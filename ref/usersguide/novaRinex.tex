%\documentclass{article}
%\usepackage{fancyvrb}
%\usepackage{perltex}
%\usepackage{xcolor}
%\usepackage{listings}
%\usepackage{longtable}
%\usepackage{multirow}
%\RecustomVerbatimEnvironment{Verbatim}{Verbatim}{frame=single}

\newcommand{\outputsize}{footnotesize}

\perlnewcommand{\getuse}[1]
{
        my $command = $_[0];
        $command = $command." > temp 2>&1";
        system("$command");

	my $output = "";
        open(input,"temp");
        while(my $line = <input>){$output = $output.$line;}
        close(input);

        return  "\\begin{\\outputsize}\n" . "\\begin{Verbatim}\n" .
                $output .
                "\\end{Verbatim}\n" . "\\end{\\outputsize}\n";
}

\perlnewcommand{\printexec}[1]
{
	my $exec = $_[0];

	return "\\begin{\\outputsize}\n" . "\\begin{Verbatim}\n" .
		$exec . "\n" .
		"\\end{Verbatim}\n" . "\\end{\\outputsize}\n";
}

\perlnewcommand{\application}[1]
{
	my $app = $_[0];
	return "\\emph{" . $app . "}";
}

%\begin{document}

\index{novaRinex!application writeup}
\section{\emph{novaRinex}}
\subsection{Overview}
The application will open and read a binary Novatel file
  (OEM2 and OEM4 receivers are supported), and convert the data to RINEX format
  observation and navigation files. The RINEX header is filled using user input
  (see below), and optional records are filled.

\subsection{Usage}

\begin{\outputsize}
\begin{longtable}{lll}
\multicolumn{3}{c}{\application{novaRinex}} \\
\multicolumn{3}{l}{\textbf{Required Arguments}} \\
\entry{Short Arg.}{Long Arg.}{Description}{1}
\entry{}{--input}{Novatel binary input file.}{1}
& & \\
\multicolumn{3}{l}{\textbf{Optional Arguments}} \\
\entry{-f}{}{Name of file containing more options ('\#' to EOL : comment).}{2}
\entry{}{--dir}{Directory in which to find input file (default ./).}{1}
\entry{}{--obs}{RINEX observation output file (RnovaRINEX.obs).}{2}
\entry{}{--nav}{RINEX navigation output file (RnovaRINEX.nav).}{2}
& & \\
\multicolumn{3}{l}{\textbf{Output RINEX Header Fields}} \\
\entry{}{--noHDopt}{If present, do not fill optional records in the output RINEX header.}{2}
\entry{}{--HDp}{Set output RINEX header 'program' field ('novaRINEX v2.1 9/07').}{2}
\entry{}{--HDr}{Set output RINEX header 'run by' field ('ARL:UT/GPSTk').}{2}
\entry{}{--HDo $<$obser$>$}{Set output RINEX header 'observer' field.}{1}
\entry{}{--HDa $<$agency$>$}{Set output RINEX header 'agency' field ('ARL:UT/GPSTk').}{2}
\entry{}{--HDm $<$marker$>$}{Set output RINEX header 'marker' field.}{1}
\entry{}{--HDn $<$number$>$}{Set output RINEX header 'number' field.}{1}
\entry{}{--HDrn $<$number$>$}{Set output RINEX header 'Rx number' field.}{1}
\entry{}{--HDrt $<$type$>$}{Set output RINEX header 'Rx type' field ('Novatel').}{2}
\entry{}{--HDrv $<$vers$>$}{Set output RINEX header 'Rx version' field ('OEM2/4').}{2}
\entry{}{--HDan $<$number$>$}{Set output RINEX header 'antenna number' field.}{1}
\entry{}{--HDat $<$type$>$}{Set output RINEX header 'antenna type' field.}{1}
\entry{}{--HDc $<$comment$>$}{Add comment to output RINEX header (>1 allowed).}{2}
& & \\
\multicolumn{3}{l}{\textbf{Output RINEX Observation Data}} \\
\entry{}{--obstype $<$OT$>$}{    Output this RINEX (standard) obs type (i.e. $<$OT$>$ is one of
                     L1,L2,C1,P1,P2,D1,D2,S1,or S2); repeat for each type.
                     NB default is ALL std. types that have data.}{4}
& & \\
\multicolumn{3}{l}{\textbf{Output Configuration}} \\
\entry{}{--begin $<$arg$>$}{Start time, arg is of the form YYYY,MM,DD,HH,Min,Sec.}{2}
\entry{}{--beginGPS $<$arg$>$}{Start time, arg is of the form GPSweek,GPSsow.}{1}
\entry{}{--end $<$arg$>$}{End time, arg is of the form YYYY,MM,DD,HH,Min,Sec.}{2}
\entry{}{--endGPS $<$arg$>$}{End time, arg is of the form GPSweek,GPSsow}{1}
\entry{}{--week $<$week$>$}{ GPS Week number of this data, NB: this is for OEM2;
                     this command serves two functions, resolving the ambiguity
                     in the 10-bit week (default uses --begin, --end, or the
                     current system time) and ensuring that ephemeris records
                     that precede any obs records are not lost.}{6}
\entry{}{--debias}{Remove an initial bias from the phase}{1}
\entry{-h}{--help}{Print this message and quit.}{1}
\entry{}{--verbose}{Print more information.}{1}
\entry{-d}{--debug}{Print extended output info.}{1}
\end{longtable}
\end{\outputsize}

\subsection{Notes}
Input is on the command line, or of the same format in a file (-f$<$file$>$).

%\end{document}
