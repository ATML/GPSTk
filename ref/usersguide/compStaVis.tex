%\documentclass{article}
%\usepackage{fancyvrb}
%\usepackage{perltex}
%\usepackage{xcolor}
%\usepackage{listings}
%\usepackage{longtable}
%\usepackage{multirow}
%\RecustomVerbatimEnvironment{Verbatim}{Verbatim}{frame=single}

\newcommand{\outputsize}{footnotesize}

\perlnewcommand{\getuse}[1]
{
        my $command = $_[0];
        $command = $command." > temp 2>&1";
        system("$command");

	my $output = "";
        open(input,"temp");
        while(my $line = <input>){$output = $output.$line;}
        close(input);

        return  "\\begin{\\outputsize}\n" . "\\begin{Verbatim}\n" .
                $output .
                "\\end{Verbatim}\n" . "\\end{\\outputsize}\n";
}

\perlnewcommand{\printexec}[1]
{
	my $exec = $_[0];

	return "\\begin{\\outputsize}\n" . "\\begin{Verbatim}\n" .
		$exec . "\n" .
		"\\end{Verbatim}\n" . "\\end{\\outputsize}\n";
}

\perlnewcommand{\application}[1]
{
	my $app = $_[0];
	return "\\emph{" . $app . "}";
}

%\begin{document}
\index{compStaVis!application writeup}
\section{\emph{compStaVis}}
\subsection{Overview}
This application computes station visibility.
\subsection{Usage}
\begin{\outputsize}
\begin{longtable}{lll}
\multicolumn{3}{c}{\application{compStaVis}} \\
\multicolumn{3}{l}{\textbf{Required Arguments}} \\
\entry{Short Arg.}{Long Arg.}{Description}{1}
\entry{-o}{--output-file=ARG}{Name of the output file to write.}{1}
\entry{-n}{--nav=ARG}{Name of navigation file.}{1}
\entry{-c}{--mscfile=ARG}{Name of MS coordinates file.}{1}
& & \\
\multicolumn{3}{l}{\textbf{Optional Arguments}} \\
\entry{Short Arg.}{Long Arg.}{Description}{1}
\entry{-d}{--debug}{Increase debug level.}{1}
\entry{-v}{--verbose}{Increase verbosity.}{1}
\entry{-h}{--help}{Print help usage.}{1}
\entry{-p}{--int=ARG}{Interval in seconds.}{1}
\entry{-e}{--minelv=ARG}{Minimum elevation angle.}{1}
\entry{-t}{--navFileType=ARG}{FALM, FEPH, RNAV, YUMA, SEM, or SP3.}{1}
\entry{-m}{--max-SV=ARG}{Maximum number of SVs tracked simultaneously.}{2}
\entry{-D}{--detail}{Pritn SV count for each interval.}{1}
\entry{-x}{--exclude=ARG}{Exclude station.}{1}
\entry{-i}{--include=ARG}{Include station.}{1}
\entry{-s}{--start-time=TIME}{Start time of evaluation ("m/d/y H:M").}{1}
\entry{-z}{--end-time=TIME}{End time of evaluation ("m/d/y H:M").}{1}
\end{longtable}
\end{\outputsize}
%%\end{document}
