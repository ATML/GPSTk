%\documentclass{article}
%\usepackage{fancyvrb}
%\usepackage{src/perltex}
%\usepackage{src/xcolor}
%\usepackage{listings}
%\usepackage{longtable}
%\usepackage{multirow}
%\RecustomVerbatimEnvironment{Verbatim}{Verbatim}{frame=single}

\newcommand{\outputsize}{footnotesize}

\perlnewcommand{\getuse}[1]
{
        my $command = $_[0];
        $command = $command." > temp 2>&1";
        system("$command");

	my $output = "";
        open(input,"temp");
        while(my $line = <input>){$output = $output.$line;}
        close(input);

        return  "\\begin{\\outputsize}\n" . "\\begin{Verbatim}\n" .
                $output .
                "\\end{Verbatim}\n" . "\\end{\\outputsize}\n";
}

\perlnewcommand{\printexec}[1]
{
	my $exec = $_[0];

	return "\\begin{\\outputsize}\n" . "\\begin{Verbatim}\n" .
		$exec . "\n" .
		"\\end{Verbatim}\n" . "\\end{\\outputsize}\n";
}

\perlnewcommand{\application}[1]
{
	my $app = $_[0];
	return "\\emph{" . $app . "}";
}

%\begin{document}

\index{navsum!application writeup}

\section{\emph{navsum}}
\subsection{Overview}
The application prints the contents of an FIC or RINEX file into a human readable format and allows for the filtering of the data.

\subsection{Usage}
\begin{\outputsize}
\begin{longtable}{lll}
\multicolumn{3}{c}{\application{navsum}} \\
\multicolumn{3}{l}{\textbf{Required Arguments}} \\
\entry{Short Arg.}{Long Arg.}{Description}{1}
\entry{-i}{--input=ARG}{Name of an input navigation message file}{1}
\entry{-o}{--output=ARG}{Name of an output file}{1}
& & \\

\multicolumn{3}{l}{\textbf{Optional Arguments}} \\
\entry{Short Arg.}{Long Arg.}{Description}{1}
\entry{-d}{--debug}{Increase debug level}{1}
\entry{-v}{--verbose}{Increase verbosity}{1}
\entry{-h}{--help}{Print help usage}{1}
\entry{-a}{--all-records}{Unless otherwise specified, use default values for record filtration}{2}
\entry{-t}{--time=TIME}{Start time (of data) for processing}{1}
\entry{-e}{--end-time=TIME}{End time (of data) for processing}{1}
\entry{-p}{--prn=NUM}{PRN(s) to include}{1}
\entry{-b}{--block=NUM}{FIC block number(s) to process ((9)109 (Engineering) ephemerides, (62)162 (engineering) almanacs)}{3}
\entry{-r}{--RINEX}{Assume input file is a RINEX navigation message file}{2}
\end{longtable}
\end{\outputsize}

\subsection{Examples}

\subsection{Notes}

%\end{document}
