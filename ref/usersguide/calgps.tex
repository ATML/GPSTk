%\documentclass{article}
%\usepackage{fancyvrb}
%\usepackage{src/perltex}
%\usepackage{src/xcolor}
%\usepackage{listings}
%\usepackage{longtable}
%\usepackage{multirow}
%\RecustomVerbatimEnvironment{Verbatim}{Verbatim}{frame=single}

\newcommand{\outputsize}{footnotesize}

\perlnewcommand{\getuse}[1]
{
        my $command = $_[0];
        $command = $command." > temp 2>&1";
        system("$command");

	my $output = "";
        open(input,"temp");
        while(my $line = <input>){$output = $output.$line;}
        close(input);

        return  "\\begin{\\outputsize}\n" . "\\begin{Verbatim}\n" .
                $output .
                "\\end{Verbatim}\n" . "\\end{\\outputsize}\n";
}

\perlnewcommand{\printexec}[1]
{
	my $exec = $_[0];

	return "\\begin{\\outputsize}\n" . "\\begin{Verbatim}\n" .
		$exec . "\n" .
		"\\end{Verbatim}\n" . "\\end{\\outputsize}\n";
}

\perlnewcommand{\application}[1]
{
	my $app = $_[0];
	return "\\emph{" . $app . "}";
}

%\begin{document}

\index{calgps!application writeup}
\section{\emph{calgps}}
\subsection{Overview}
This application generates a dual GPS and Julian calendar. The arguments and 
format are inspired by the UNIX 'cal' utility. With no arguments, the current 
argument is printed. The last and next month can also be printed. Also, the 
current or any given year can be printed.
\subsection{Usage}
\begin{\outputsize}
\begin{longtable}{lll}
\multicolumn{3}{l}{\textbf{Optional Arguments}} \\
Short Arg. & Long Arg. & Description \\
-h & --help & Generates help output. \\ 
-3 & --three-months & Prints a GPS calendar for the previous, current, and next month. \\
-y & --year & Prints a GPS calendar for the entire current year. \\
-Y year & --specific-year=NUM & Prints a GPS calendar for the entire specified year.
\end{longtable}
\end{\outputsize}

\subsection{Examples}

\getuse{calgps -3}

\getuse{calgps -Y 1998}

\subsection{Notes}
If multiple options are given only the first is considered.

%\end{document}

