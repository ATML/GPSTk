\documentclass{article}
\usepackage{fancyvrb}
\usepackage{perltex}
\usepackage{xcolor}
\usepackage{listings}
\usepackage{multirow}
\usepackage{longtable}
\RecustomVerbatimEnvironment{Verbatim}{Verbatim}{frame=single}

\newcommand{\outputsize}{footnotesize}

\perlnewcommand{\getuse}[1]
{
        my $command = $_[0];
        $command = $command." > temp 2>&1";
        system("$command");

	my $output = "";
        open(input,"temp");
        while(my $line = <input>){$output = $output.$line;}
        close(input);

        return  "\\begin{\\outputsize}\n" . "\\begin{Verbatim}\n" .
                $output .
                "\\end{Verbatim}\n" . "\\end{\\outputsize}\n";
}

\perlnewcommand{\printexec}[1]
{
	my $exec = $_[0];

	return "\\begin{\\outputsize}\n" . "\\begin{Verbatim}\n" .
		$exec . "\n" .
		"\\end{Verbatim}\n" . "\\end{\\outputsize}\n";
}

\perlnewcommand{\application}[1]
{
	my $app = $_[0];
	return "\\emph{" . $app . "}";
}

\begin{document}

\index{EditRinex!application writeup}
\section{\emph{EditRinex}}
\subsection{Overview}
This application will open and read one RINEX file, apply editing commands,
 and write the modified RINEX data to another RINEX file(s).
 Input is on the command line, or of the same format in a file (-f$<$file$>$).

\subsection{Usage}
\begin{\outputsize}
\begin{longtable}{lll}
\multicolumn{3}{c}{\application{EditRinex}} \\
\multicolumn{3}{l}{\textbf{Optional Arguments}} \\
\entry{Short Arg.}{Long Arg.}{Description}{1}
\entry{-f}{--file $<$file$>$}{File containing more options.}{1}
\entry{-l}{--log $<$file$>$}{Output log file name.}{1}
\entry{-h}{--help}{Print syntax and quit.}{1}
\entry{-d}{--debug}{Print extended output info.}{1}
\entry{-v}{--verbose}{Print extended output info.}{1}
\entry{}{$<$REC$>$}{Rinex editing commands - following:}{1}\\
\end{longtable}
\end{\outputsize}

\subsubsection{Rinex Editor Commands}
 Commands consist of an identifier and a comma-delimited data field; they may be
 separated by space(s) `--id $<$data$>$' (two minuses) or not `-id$<$data$>$' (one minus).

 Examples are `--IF myFile' or `-IFmyFile'; `--HDc msg' or `--HD cmsg' or `-HDcmsg';
 --BZ or -BZ; `--DD +$<$SV,OT,t$>$' or `--DD+ $<$SV,OT,t$>$' or `-DD+$<$SV,OT,t$>$'.

 The data field contains no whitespace and sub-fields are comma-delimited.
 $<$SV$>$ is a RINEX `system and id' identifier, e.g. G27 (= GPS PRN 27);
   satellite system alone denotes `all satellites this system', e.g. `R' (GLONASS).
 $<$OT$>$ is a RINEX observation type, e.g. L1 or P2, and is case sensitive.
 $<$time$>$ is either $<$GPSweek,GPSsecOfWeek$>$ or $<$year,mon,day,hour,min,second$>$.
\\
\begin{\outputsize}
 File I/O:\\
 ---------\\
 -IF$<$file$>$       Input RINEX observation file name [may be repeated] (required). \\
 -ID$<$dir$>$        Directory in which to find input file. \\
 -OF$<$file$>$       Output RINEX file name (required, or -OF$<$file$>$,$<$time$>$). \\
 -OF$<$f$>$,$<$time$>$   At RINEX epoch $<$time$>$, close output file and open another named $<$f$>$. \\
 -OD$<$dir$>$        Directory in which to put output file(s). \\
\\
 Output RINEX header:\\
 --------------------\\
 -HDf            If present, fill optional records in the output RINEX header\\
 -HDp$<$program$>$   Set output RINEX header `program' field.\\
 -HDr$<$run\_by$>$    Set output RINEX header `run by' field.\\
 -HDo$<$observer$>$  Set output RINEX header `observer' field.\\
 -HDa$<$agency$>$    Set output RINEX header `agency' field.\\
 -HDx$<$x,y,z$>$     Set output RINEX header `position' field to ECEF position (x,y,z).\\
 -HDm$<$marker$>$    Set output RINEX header `marker' field.\\
 -HDn$<$number$>$    Set output RINEX header `number' field.\\
 -HDc$<$comment$>$   Add comment to output RINEX header (more than one allowed).\\
 -HDdc           Delete all comments in output RINEX header.\\
           (NB -HDdc cannot delete comments created by *subsequent* -HDc commands).\\
\\
 Output RINEX observation types (also see `Specific edit commands' below):\\
 -------------------------------------------------------------------------\\
 -AO$<$OT$>$         Add observation type OT to header and observation data.\\
 -DO$<$OT$>$         Delete observation type OT entirely (including in header).\\
\\
 Time-related edit commands:\\
 ---------------------------\\
 -TB$<$time$>$       Begin time: reject data before this time (also used for decimation).\\
 -TE$<$time$>$       End   time: reject data after this time.\\
 -TT$<$dt$>$         Tolerance in comparing times, in seconds (default=1ms).\\
 -TN$<$dt$>$         Decimate data to epochs = Begin + integer*dt (within tolerance).\\
\\
 Specific edit commands:\\
 -----------------------\\
 (Generally each `+' command (e.g DA+$<$time$>$) has a corresponding `-' command,
     and vice-versa; if not, end-of-file or beginning-of-file is assumed.
  Note that one-time commands are applied AFTER other commands of the same type.)\\
\\
 Delete commands:\\
 -DA+$<$time$>$      Delete all data beginning at this time.\\
 -DA-$<$time$>$      Stop deleting data at this time.\\
 -DO$<$OT$>$         Delete observation type OT entirely (including in header).\\
 -DS$<$SV$>$         Delete all data for satellite SV entirely (SV may be system only).\\
 -DS$<$SV$>$,$<$time$>$  Delete all data for satellite SV at this single time only.\\
 -DS+$<$SV$>$,$<$time$>$ Delete all data for satellite SV beginning at this time.\\
 -DS-$<$SV$>$,$<$time$>$ Stop deleting all data for satellite SV at this time.\\
 -DD$<$SV,OT,t$>$    Delete a single RINEX datum(SV,OT,t) at time $<$t$>$.\\
 -DD+$<$SV,OT,t$>$   Delete all (SV,OT) data, beginning at time $<$t$>$.\\
 -DD-$<$SV,OT,t$>$   Stop deleting all (SV,OT) data at time $<$t$>$.\\
     (Deleting data for one OT means setting it to zero - as RINEX requires.)\\
\\
 Set commands:\\
 -SD$<$SV,OT,t,d$>$  Set data(SV,OT,t) to $<$d$>$ at time $<$t$>$.\\
 -SS$<$SV,OT,t,s$>$  Set ssi(SV,OT,t) to $<$s$>$ at time $<$t$>$.\\
 -SL+$<$SV,OT,t,l$>$ Set all lli(SV,OT,t) to $<$l$>$ at time $<$t$>$.\\
 -SL-$<$SV,OT,t,l$>$ Stop setting lli(SV,OT,t) to $<$l$>$ at time $<$t$>$ (`,$<$l$>$' is optional).\\
 -SL$<$SV,OT,t,l$>$  Set lli(SV,OT,t) to $<$l$>$ at the single time $<$t$>$ only.\\
\\
 Bias commands:\\
  (BD commands apply only when data is non-zero, unless -BZ appears.)\\
 -BZ             Apply BD commands even when data is zero (i.e. `missing').\\
 -BD$<$SV,OT,t,d$>$  Add the value of $<$d$>$ to data(SV,OT,t) at time $<$t$>$.\\
 -BD+$<$SV,OT,t,d$>$ Add value $<$d$>$ to data(SV,OT) beginning at time $<$t$>$.\\
 -BD-$<$SV,OT,t,d$>$ Stop adding $<$d$>$ to data(SV,OT) at time $<$t$>$ (`,$<$d$>$' optional).\\
 -BS$<$SV,OT,t,s$>$  Add the value of $<$s$>$ to ssi(SV,OT,t) at time $<$t$>$.\\
 -BL$<$SV,OT,t,l$>$  Add the value of $<$l$>$ to lli(SV,OT,t) at time <t$>$.\\
\end{\outputsize}

\subsection{Examples}
\subsubsection{Changing the APPROX position in the file acor1480.08o to the center of the Earth. Writes a new file called acor1480.08o.mod}
\begin{verbatim}
user@host:~$ EditRinex -IFacor1480.08o -OFacor1480.08o.mod -HDx0,0,0 
\end{verbatim}

\subsubsection{Removing a satelite, PRN 29, from an observation file, onsa2240.05o. Creates a new file, temp.o}
\begin{verbatim}
EditRinex -IFonsa2240.05o -OFtemp.o 
\end{verbatim}
\end{document}
