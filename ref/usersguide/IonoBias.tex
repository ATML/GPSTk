%\documentclass{article}
%\usepackage{fancyvrb}
%\usepackage{perltex}
%\usepackage{xcolor}
%\usepackage{listings}
%\usepackage{longtable}
%\usepackage{multirow}
%\RecustomVerbatimEnvironment{Verbatim}{Verbatim}{frame=single}

\newcommand{\outputsize}{footnotesize}

\perlnewcommand{\getuse}[1]
{
        my $command = $_[0];
        $command = $command." > temp 2>&1";
        system("$command");

	my $output = "";
        open(input,"temp");
        while(my $line = <input>){$output = $output.$line;}
        close(input);

        return  "\\begin{\\outputsize}\n" . "\\begin{Verbatim}\n" .
                $output .
                "\\end{Verbatim}\n" . "\\end{\\outputsize}\n";
}

\perlnewcommand{\printexec}[1]
{
	my $exec = $_[0];

	return "\\begin{\\outputsize}\n" . "\\begin{Verbatim}\n" .
		$exec . "\n" .
		"\\end{Verbatim}\n" . "\\end{\\outputsize}\n";
}

\perlnewcommand{\application}[1]
{
	my $app = $_[0];
	return "\\emph{" . $app . "}";
}

%\begin{document}

\index{IonoBias!application writeup}

\section{\emph{IonoBias}}
\subsection{Overview}
The application will open and read several preprocessed RINEX obs files (containing obs types EL,LA,LO,SR or SS) and use the data to estimate satellite and receiver biases and to compute a simple ionospheric model using least squares and the slant TEC values.

\subsection{Usage}
\begin{\outputsize}
\begin{longtable}{lll}
\multicolumn{3}{c}{\application{IonoBias}} \\
\multicolumn{3}{l}{\textbf{Required Arguments}} \\
\entry{Short Arg.}{Long Arg.}{Description}{1}
\entry{}{--input}{Input RINEX obs file name(s).}{1}
& & \\
\multicolumn{3}{l}{\textbf{Optional Arguments}} \\
\entry{Short Arg.}{Long Arg.}{Description}{1}
\entry{-f}{}{File containing more options}{1}
\entry{}{--inputdir}{Path for input file(s).}{1}
& & \\
\multicolumn{3}{l}{\textbf{Ephemeris Input}} \\
\entry{Short Arg.}{Long Arg.}{Description}{1}
\entry{}{--navdir}{Path of navigation file(s).}{1}
\entry{}{--nav}{Navigation (RINEX (nav) OR SP3) file(s).}{1}
& & \\
\multicolumn{3}{l}{\textbf{Output}} \\
\entry{Short Arg.}{Long Arg.}{Description}{1}
\entry{}{--datafile}{Data (AT) file name, for output and/or input.}{1}
\entry{}{--log}{Output log file name.}{1}
\entry{}{--biasout}{Output satellite+receiver biases file name.}{1}
& & \\
\multicolumn{3}{l}{\textbf{Time Limits}} \\
\entry{Short Arg.}{Long Arg.}{Description}{1}
\entry{}{--BeginTime}{Start time, arg is of the form YYYY,MM,DD,HH,Min,Sec.}{2}
\entry{}{--BeginGPSTime}{Start time, arg is of the form GPSweek,GPSsow.}{2}
\entry{}{--EndTime}{End time, arg is of the form YYYY,MM,DD,HH,Min,Sec.}{2}
\entry{}{--EndGPSTime}{End time, arg is of the form GPSweek,GPSsow.}{2}
& & \\
\multicolumn{3}{l}{\textbf{Processing}} \\
\entry{Short Arg.}{Long Arg.}{Description}{1} 
\entry{}{--NoEstimation}{Do NOT perform the estimation (default=false).}{1}
\entry{}{--NoPreprocess}{Skip preprocessing; read (existing) AT file (false).}{2}
\entry{}{--NoSatBiases}{Compute Receiver biases ONLY (not Rx+Sat biases) (false).}{2}
\entry{}{--Model}{Ionospheric model: type is linear, quadratic or cubic.}{2}
\entry{}{--MinPoints}{Minimum points per satellite required.}{1}
\entry{}{--MinTimeSpan}{Minimum timespan per satellite required (minutes).}{2}
\entry{}{--MinElevation}{Minimum elevation angle (degrees).}{1}
\entry{}{--MinLatitude}{Minimum latitude (degrees).}{1}
\entry{}{--MaxLatitude}{Maximum latitude (degrees).}{1}
\entry{}{--MinLongitude}{Minimum longitude (degrees).}{1}
\entry{}{--MaxLongitude}{Maximum longitude (degrees).}{1}
\entry{}{--TimeSector}{Time sector (day | night | both).}{1}
\entry{}{--TerminOffset}{Terminator offset (minutes).}{1}
\entry{}{--IonoHeight}{Ionosphere height (km).}{1}
& & \\
\multicolumn{3}{l}{\textbf{Other Options}} \\
\entry{Short Arg.}{Long Arg.}{Description}{1}
\entry{}{--XSat}{Exclude this satellite ($<$sat$>$ may be $<$system$>$ only).}{2}
\entry{-v}{--verbose}{Print extended output info.}{1}
\entry{-d}{--debug}{Print extended output info.}{1}
\entry{-h}{--help}{Print syntax and quit.}{1}

\end{longtable}
\end{\outputsize}

\subsection{Examples}
\begin{\outputsize}
\begin{lstlisting}
> IonoBias --inputdir data_set --navdir data_set --input s081213a.99o --input s081214a.99o
--input s081215a.99o --nav s081213a.99n --nav s081214a.99n --nav s081215a.99n --datafile output}
IonoBias, built on the GPSTK ToolKit, Ver 1.0 6/25/04, Run 2006/08/17 09:50:59
IonoBias output directed to log file IonoBias.log
IonoBias timing: 6.210 seconds.
\end{lstlisting}

\begin{verbatim}
Output File Snippet

    3     3 Number (max, good) stations in this file
010101101100001111110111011101110
010101101100001111110111011101110
010100101100001111110111011101110
Npt  9737 Sta 85408 LLH    30.2160   262.2746   163.4226
1021      0.0   0.00000 -463513.64930 0.32    0.000      1  1   1
1021      0.0   0.00000 -463513.64930 0.32    0.000      1 14   1
1021      0.0   0.00000 -463513.64930 0.32    0.000      1 15   1
1021      0.0   0.00000 -463513.64930 0.32    0.000      1 21   1
1021      0.0   0.00000 -463513.64930 0.32    0.000      1 22   1
1021      0.0   0.00000 -463513.64930 0.32    0.000      1 25   1
1021      0.0   0.00000 -463513.64930 0.32    0.000      1 29   1
1021      0.0   0.00000 -463513.64930 0.32    0.000      1 30   1
1021     30.0   0.00000 -463513.52430 0.32    0.000      1  1   1
1021     30.0   0.00000 -463513.52430 0.32    0.000      1 14   1

\end{verbatim}

\end{\outputsize}

\subsection{Notes}
Input can be either on the command line or put in a file and then input using the -f option. The file is formatted just as if
it were the command line.

%\end{document}

