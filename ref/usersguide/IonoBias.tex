%\documentclass{article}
%\usepackage{fancyvrb}
%\usepackage{src/perltex}
%\usepackage{src/xcolor}
%\usepackage{listings}
%\usepackage{longtable}
%\usepackage{multirow}
%\RecustomVerbatimEnvironment{Verbatim}{Verbatim}{frame=single}

\newcommand{\outputsize}{footnotesize}

\perlnewcommand{\getuse}[1]
{
        my $command = $_[0];
        $command = $command." > temp 2>&1";
        system("$command");

	my $output = "";
        open(input,"temp");
        while(my $line = <input>){$output = $output.$line;}
        close(input);

        return  "\\begin{\\outputsize}\n" . "\\begin{Verbatim}\n" .
                $output .
                "\\end{Verbatim}\n" . "\\end{\\outputsize}\n";
}

\perlnewcommand{\printexec}[1]
{
	my $exec = $_[0];

	return "\\begin{\\outputsize}\n" . "\\begin{Verbatim}\n" .
		$exec . "\n" .
		"\\end{Verbatim}\n" . "\\end{\\outputsize}\n";
}

\perlnewcommand{\application}[1]
{
	my $app = $_[0];
	return "\\emph{" . $app . "}";
}

%\begin{document}

\index{IonoBias!application writeup}

\section{\emph{IonoBias}}
\subsection{Overview}
The application will open and read several preprocessed Rinex obs files
 (containing obs types EL,LA,LO,SR or SS) and use the data to estimate
 satellite and receiver biases and to compute a simple ionospheric model
 using least squares and the slant TEC values.

\subsection{Usage}
\begin{\outputsize}
\begin{longtable}{lll}
\multicolumn{3}{c}{\application{IonoBias}} \\
\multicolumn{3}{l}{\textbf{Optional Arguments}} \\
\entry{}{--input}{Input Rinex obs file name(s)}{1}
& & \\
\multicolumn{3}{l}{\textbf{Optional Arguments}} \\
\entry{Short Arg.}{Long Arg.}{Description}{1}
\entry{-f}{}{file containing more options}{1}
\entry{}{--inputdir}{Path for input file(s)}{1}
& & \\
\multicolumn{3}{l}{\textbf{Ephemeris input}} \\
\entry{}{--navdir}{Path of navigation file(s)}{1}
\entry{}{--nav}{Navigation (Rinex Nav OR SP3) file(s)}{1}
& & \\
\multicolumn{3}{l}{\textbf{Output}} \\
\entry{}{--datafile}{Data (AT) file name, for output and/or input}{1}
\entry{}{--log}{Output log file name}{1}
\entry{}{--biasout}{Output satellite+receiver biases file name}{1}
& & \\
\multicolumn{3}{l}{\textbf{Time limits}} \\
\entry{}{--BeginTime}{Start time, arg is of the form YYYY,MM,DD,HH,Min,Sec}{2}
\entry{}{--BeginGPSTime}{Start time, arg is of the form GPSweek,GPSsow}{1}
\entry{}{--EndTime}{End time, arg is of the form YYYY,MM,DD,HH,Min,Sec}{2}
\entry{}{--EndGPSTime}{End time, arg is of the form GPSweek,GPSsow}{1}
& & \\
\multicolumn{3}{l}{\textbf{Processing}} \\
\entry{}{--NoEstimation}{Do NOT perform the estimation (default=false).}{1}
\entry{}{--NoPreprocess}{Skip preprocessing; read (existing) AT file (false).}{2}
\entry{}{--NoSatBiases}{Compute Receiver biases ONLY (not Rx+Sat biases) (false).}{2}
\entry{}{--Model}{Ionospheric model: type is linear, quadratic or cubic}{2}
\entry{}{--MinPoints}{Minimum points per satellite required}{1}
\entry{}{--MinTimeSpan}{Minimum timespan per satellite required (minutes)}{2}
\entry{}{--MinElevation}{Minimum elevation angle (degrees)}{1}
\entry{}{--MinLatitude}{Minimum latitude (degrees)}{1}
\entry{}{--MaxLatitude}{Maximum latitude (degrees)}{1}
\entry{}{--MinLongitude}{Minimum longitude (degrees)}{1}
\entry{}{--MaxLongitude}{Maximum longitude (degrees)}{1}
\entry{}{--TimeSector}{Time sector (day | night | both)}{1}
\entry{}{--TerminOffset}{Terminator offset (minutes)}{1}
\entry{}{--IonoHeight}{Ionosphere height (km)}{1}
& & \\
\multicolumn{3}{l}{\textbf{Other options}} \\
\entry{}{--XSat}{Exclude this satellite (<sat> may be <system> only)}{2}
\entry{-v}{--verbose}{print extended output info.}{1}
\entry{-d}{--debug}{print extended output info.}{1}
\entry{-h}{--help}{print syntax and quit.}{1}

\end{longtable}
\end{\outputsize}

\subsection{Examples}

\subsection{Notes}
Input is on the command line, or of the same format in a file (-f$<$file$>$).

%\end{document}

