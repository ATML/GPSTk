%\documentclass{article}
%\usepackage{fancyvrb}
%\usepackage{perltex}
%\usepackage{xcolor}
%\usepackage{listings}
%\usepackage{longtable}
%\usepackage{multirow}
%\RecustomVerbatimEnvironment{Verbatim}{Verbatim}{frame=single}

\newcommand{\outputsize}{footnotesize}

\perlnewcommand{\getuse}[1]
{
        my $command = $_[0];
        $command = $command." > temp 2>&1";
        system("$command");

	my $output = "";
        open(input,"temp");
        while(my $line = <input>){$output = $output.$line;}
        close(input);

        return  "\\begin{\\outputsize}\n" . "\\begin{Verbatim}\n" .
                $output .
                "\\end{Verbatim}\n" . "\\end{\\outputsize}\n";
}

\perlnewcommand{\printexec}[1]
{
	my $exec = $_[0];

	return "\\begin{\\outputsize}\n" . "\\begin{Verbatim}\n" .
		$exec . "\n" .
		"\\end{Verbatim}\n" . "\\end{\\outputsize}\n";
}

\perlnewcommand{\application}[1]
{
	my $app = $_[0];
	return "\\emph{" . $app . "}";
}

%\begin{document}

\index{DOPcalc!application writeup}
\section{\emph{DOPcalc}}
\subsection{Overview}
This application computes position, time, and geometric dilution of precision (DOP) parameters.

\subsection{Usage}
\begin{\outputsize}
\begin{longtable}{lll}
\multicolumn{3}{c}{\application{DOPcalc}} \\
\multicolumn{3}{l}{\textbf{Required Arguments}} \\
\entry{Short Arg.}{Long Arg.}{Description}{1}
\entry{-e}{--eph=ARG}{Where to get the ephemeris data.  Acceptable formats include RINEX nav, FIC, MDP, SP3, YUMA, and SEM.  Repeat for multiple files.}{3}
\entry{-o}{--obs=ARG}{Where to get the observation data.  Acceptable formats include RINEX obs, MDP, smooth, Novatel, and raw Ashtech.  Repeat for multiple files.  If a RINEX obs file is provided, the position will be taken from the header unless otherwise specified.}{5}
& & \\
\multicolumn{3}{l}{\textbf{Optional Arguments}} \\
\entry{Short Arg.}{Long Arg.}{Description}{1}
\entry{-d}{--debug}{Increase debug level.}{1}
\entry{-v}{--verbose}{Increase verbosity.}{1}
\entry{-h}{--help}{Print help usage.}{1}
\entry{-p}{--position=ARG}{User position in ECEF (x,y,z) coordinates.  Format as a string: "X Y Z".}{2}
\entry{}{--el-mask=ARG}{Elevation mask to apply, in degrees.  The default is 0.}{2}
\entry{-c}{--msc=ARG}{Station coordinate file.}{1}
\entry{-m}{--msid=ARG}{Monitor station ID number.}{1}
\end{longtable}
\end{\outputsize}
%\end{document}
