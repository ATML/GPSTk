%\documentclass{article}
%\usepackage{fancyvrb}
%\usepackage{perltex}
%\usepackage{xcolor}
%\usepackage{listings}
%\usepackage{longtable}
%\usepackage{multirow}
%\RecustomVerbatimEnvironment{Verbatim}{Verbatim}{frame=single}

\newcommand{\outputsize}{footnotesize}

\perlnewcommand{\getuse}[1]
{
        my $command = $_[0];
        $command = $command." > temp 2>&1";
        system("$command");

	my $output = "";
        open(input,"temp");
        while(my $line = <input>){$output = $output.$line;}
        close(input);

        return  "\\begin{\\outputsize}\n" . "\\begin{Verbatim}\n" .
                $output .
                "\\end{Verbatim}\n" . "\\end{\\outputsize}\n";
}

\perlnewcommand{\printexec}[1]
{
	my $exec = $_[0];

	return "\\begin{\\outputsize}\n" . "\\begin{Verbatim}\n" .
		$exec . "\n" .
		"\\end{Verbatim}\n" . "\\end{\\outputsize}\n";
}

\perlnewcommand{\application}[1]
{
	my $app = $_[0];
	return "\\emph{" . $app . "}";
}

%\begin{document}

\index{TECMaps!application writeup}

\section{\emph{TECMaps}}
\subsection{Overview}
Program TECMaps reads RINEX data files containing extended RINEX observation types EL, AZ and SR or VR from several sites and at each epoch fits the vertical TEC data to a model of the ionosphere on a two-dimensional grid surface. Hardware TEC measurement biases are corrected, using input from the program IonoBias. The user can specify the type of grid, the type of TEC data and the model to be used. Output is in the form of files, one per epoch, which can be used to plot the 2D ionospheric TEC surface.
\subsection{Usage}
\begin{\outputsize}
\begin{longtable}{lll}
\multicolumn{3}{c}{\application{TECMaps}} \\
\multicolumn{3}{l}{\textbf{Required Arguments}} \\
\entry{Short Arg.}{Long Arg.}{Description}{1}
\entry{}{--input}{Input RINEX obs file name(s).}{1}

& & \\
\multicolumn{3}{l}{\textbf{Optional Arguments}} \\
\entry{Short Arg.}{Long Arg.}{Description}{1}
\entry{-f}{}{File containing more options.}{1}
& & \\
\multicolumn{3}{l}{\textbf{Reference Station Position (One Required)}} \\
\entry{Short Arg.}{Long Arg.}{Description}{1}
\entry{}{--RxLLH $<$l,l,h$>$}{Reference site position in geodetic lat, lon (E), ht (deg,deg,m).}{2}
\entry{}{--RxXYZ $<$x,y,z$>$}{Reference site position in ECEF coordinates (m).}{2}
\entry{}{--inputdir}{Path for input file(s).}{1}
& & \\
\multicolumn{3}{l}{\textbf{Ephemeris Input}} \\
\entry{Short Arg.}{Long Arg.}{Description}{1}
\entry{}{--navdir}{Path of navigation file(s).}{1}
\entry{}{--nav}{Navigation (RINEX navigation OR SP3) file(s).}{1}
& & \\
\multicolumn{3}{l}{\textbf{Output}} \\
\entry{Short Arg.}{Long Arg.}{Description}{1}
\entry{}{--log}{Output log file name.}{1}
& & \\
\multicolumn{3}{l}{\textbf{Time Limits}} \\
\entry{Short Arg.}{Long Arg.}{Description}{1}
\entry{}{--BeginTime}{Start time, arg is of the form YYYY,MM,DD,HH,Min,Sec.}{2}
\entry{}{--BeginGPSTime}{Start time, arg is of the form GPSweek,GPSsow.}{1}
\entry{}{--EndTime}{End time, arg is of the form YYYY,MM,DD,HH,Min,Sec.}{2}
\entry{}{--EndGPSTime}{End time, arg is of the form GPSweek,GPSsow.}{1}
& & \\
\multicolumn{3}{l}{\textbf{Processing}} \\
\entry{Short Arg.}{Long Arg.}{Description}{1}
\entry{}{--noVTECmap}{Do NOT create the VTEC map.}{1}
\entry{}{--MUFmap}{Create MUF map as well as VTEC map.}{1}
\entry{}{--F0F2map}{Create F0F2 map as well as VTEC map.}{1}
\entry{}{--Title1 $<$title$>$}{Title information.}{1}
\entry{}{--Title2 $<$title$>$}{Second title information.}{1}
\entry{}{--BaseName $<$name$>$}{Base name for output files.}{1}
\entry{}{--DecorrError $<$de$>$}{Decorrelation error rate in TECU/1000km (3).}{1}
\entry{}{--Biases $<$file$>$}{File containing estimated sat+rx biases (Prgm IonoBias).}{2}
\entry{}{--ElevThresh $<$ele$>$}{Minimum elevation (6 degrees).}{1}
\entry{}{--MinAcqTime$<$t$>$}{Minimum acquisition time (0 seconds).}{1}
\entry{}{--FlatFit}{Flat fit type (default).}{1}
\entry{}{--LinearFit}{Linear fit type.}{1}
\entry{}{--IonoHeight $<$n$>$}{Ionosphere height (km).}{1}
\entry{}{--Offset $<$tec$>$}{Overall bias to add to data (TECU).}{1}
& & \\
\multicolumn{3}{l}{\textbf{Grid}} \\
\entry{Short Arg.}{Long Arg.}{Description}{1}
\entry{}{--UniformSpacing}{Grid uniform in space (XYZ) (default).}{1}
\entry{}{--UniformGrid}{Grid uniform in Lat and Lon.}{1}
\entry{}{--OutputGrid}{Output the grid to file $<$basename.LL$>$.}{1}
\entry{}{--GnuplotOutput}{Write the grid file for gnuplot (default: for Matlab).}{2}
\entry{}{--NumLat $<$n$>$}{Number of latitude grid points (40).}{1}
\entry{}{--NumLon $<$n$>$}{Number of longitude grid points (40)}{1}
\entry{}{--BeginLat $<$lat$>$}{Beginning latitude (21 degrees).}{1}
\entry{}{--BeginLon $<$lon$>$}{Beginning longitude (230 degrees E).}{1}
\entry{}{--DeltaLat $<$del$>$}{Grid spacing in latitude (0.25 degrees).}{1}
\entry{}{--DeltaLon $<$del$>$}{Grid spacing in longitude (1.0 degrees).}{1}
& & \\
\multicolumn{3}{l}{\textbf{Other Options}} \\
\entry{Short Arg.}{Long Arg.}{Description}{1}
\entry{}{--XSat}{Exclude this satellite ($<$sat$>$ may be $<$system$>$ only).}{2}
& & \\
\multicolumn{3}{l}{\textbf{Help}} \\
\entry{Short Arg.}{Long Arg.}{Description}{1}
\entry{-v}{--verbose}{Print extended output info.}{1}
\entry{-d}{--debug}{Increase debug level.}{1}
\entry{-h}{--help}{Print syntax and summary of input, then quit.}{1}

\end{longtable}
\end{\outputsize}

\subsection{Notes}
Input is on the command line, or of the same format in a file (-f$<$file$>$).

%\end{document}

