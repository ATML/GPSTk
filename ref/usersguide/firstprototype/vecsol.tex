\chapter{VECSOL}
by Martin Vermeer
       
The vecsol application computes a  3D vector solution using dual-frequency carrier phases. A double difference algorithm is applied with properly computed weights (elevation sine weighting)  and  correlations. The  program  iterates to convergence and attempts to resolve ambiguities to integer values if close enough.  Crude  outlier  rejection  is  provided  based  on  a  triple-difference  test. Ephemeris used are either broadcast or precise (SP3). Alternatively, also P code processing is provided.

       The solution is computed using the ionosphere-free linear combination.  The ionospheric  model included in broadcast ephemeris may be used. A standard tropospheric correction is applied, or tropospheric parameters (zenith delays) may be estimated.

\section{Inputs}

\subsection{RINEX Observation Files}
The two arguments are names of RINEX observation files. They contain the observations 
collected at the two end points 1 and 2 of the baseline.  They
must contain a sufficient set of simultaneous observations to the same satellites.

\subsection{Configuration File vecsol.conf}
The file vescol.conf contains the input options for the program, one per line.

\begin{tabular}{|l|c|l|} \hline

Options & Value & Meaning \\
phase   & 1/0   & If 1, process carrier phase data (instead of P code data) \\
truecov & 1/0   & If 1, use true double difference covariances. If 0, ignore any possible correlations \\
precise & 1/0   & If 1, use precise ephemeris, if 0, use broadcast ephemeris \\
iono    & 1/0   & If 1, use the 8-parameter ionospheric model that comes with the broadcast  ephemeris (.nav) files \\
tropo   & 1/0   & If 1, estimate troposphere parameters (zenith delays relative to the standard value, which is always applied) \\
vecmode & 1/0   & If 1, solve the vector, i.e. the three co-ordinate differences between the  baseline end points. If 0, solve for the absolute co-ordinates of both end points \\
debug   & 1/0   & If 1, produce lots of gory debugging output. See the source for what it all means \\
refsat elev & number & Minimum elevation of the reference satellite used for computing inter-satellite differences.  Good initial choice: 30.0 \\
cutoff elev & number & cut-off elevation. Good initial choice: 10.0 - 20.0 \\
\hline
\end{tabular}

\subsection{Ephemeris File Lists}
The file vecsol.nav contains the names of the navigation RINEX files ("nav files", extension). Good navigation RINEX files that are globally valid can be found  from  the  CORS  website  at http://www.ngs.noaa.gov/CORS/

The file vecsol.eph contains  the  names  of  the  precise ephemeris SP3 files (extension .sp3) to be used. These should cover the time span of the observations, with time to spare on both  ends. Note  that the date in the filenames of the SP3 files is given as GPS week + weekday, not year + day of year, as in the observation and nav files.

In the .nav and .eph files, comment lines have \# in the first position.

\section{Known Limitations}
The vecsol application doesn't currently recover at all from cycle slips, so the RINEX observation files used have to be fairly clean already.
