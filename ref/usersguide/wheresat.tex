%\documentclass{article}
%\usepackage{fancyvrb}
%\usepackage{src/perltex}
%\RecustomVerbatimEnvironment{Verbatim}{Verbatim}{frame=single}

\newcommand{\outputsize}{footnotesize}

\perlnewcommand{\getuse}[1]
{
        my $command = $_[0];
        $command = $command." > temp 2>&1";
        system("$command");

	my $output = "";
        open(input,"temp");
        while(my $line = <input>){$output = $output.$line;}
        close(input);

        return  "\\begin{\\outputsize}\n" . "\\begin{Verbatim}\n" .
                $output .
                "\\end{Verbatim}\n" . "\\end{\\outputsize}\n";
}

\perlnewcommand{\printexec}[1]
{
	my $exec = $_[0];

	return "\\begin{\\outputsize}\n" . "\\begin{Verbatim}\n" .
		$exec . "\n" .
		"\\end{Verbatim}\n" . "\\end{\\outputsize}\n";
}

\perlnewcommand{\application}[1]
{
	my $app = $_[0];
	return "\\emph{" . $app . "}";
}

%\begin{document}

\index{wheresat!application writeup}
\section{\emph{wheresat}}
\subsection{Usage}
\getuse{wheresat --help}

\subsection{Examples}
\begin{\outputsize}
\begin{Verbatim}
> wheresat -b aira1720.06n -p 2 -u "918129.01 -4346070.45 803.18"
  -s "06/21/2006 17:00:00" -e "06/21/2006 20:00:00" -t 1800
\end{Verbatim}
\begin{Verbatim}
 Antenna Position:  918129  -4.34607e+06  803.18
 Navigation File:   aira1720.06n
 Start Time:        06/21/2006 17:00:00
 End Time:          06/21/2006 20:00:00
 PRN:               2

 Prn 2 Earth-fixed position and clock information:

 Date       Time(UTC)   X (meters)          Y (meters)          Z (meters)      
 ===============================================================================
 06/21/2006 18:00:00  12758891.971859      18901201.616227      -14049016.596144
 06/21/2006 18:30:00  12847888.097031      21541501.416411      -9315422.851798 
 06/21/2006 19:00:00  12843576.989405      23087218.618683      -3957280.515764 
 06/21/2006 19:30:00  12450313.769289      23516935.034029      1667186.089065  

  . . .

    Clock Correc (s)
==================
     0.000007
     0.000007
     0.000007
     0.000007

 

 Data for user reference frame:

 Date       Time(UTC)   Azimuth        Elevation      Range to SV (m)
 =====================================================================
 06/21/2006 18:00:00  130.596202      -43.242769      29627531.177821
 06/21/2006 18:30:00  118.680085      -49.681012      29983796.522429
 06/21/2006 19:00:00  102.845663      -53.888528      30169796.433699
 06/21/2006 19:30:00  84.400419       -55.459042      30197072.648367

 Calculated 4 increments for prn 2 .


\end{Verbatim}
\end{\outputsize}

\subsection{Notes}
Wheresat uses the broadcast ephemeris within the given Rinex navigation file to compute the satellite positions.

%\end{document}

