%\documentclass{article}
%\usepackage{fancyvrb}
%\usepackage{perltex}
%\usepackage{xcolor}
%\usepackage{listings}
%\usepackage{longtable}
%\usepackage{multirow}
%\RecustomVerbatimEnvironment{Verbatim}{Verbatim}{frame=single}

\newcommand{\outputsize}{footnotesize}

\perlnewcommand{\getuse}[1]
{
        my $command = $_[0];
        $command = $command." > temp 2>&1";
        system("$command");

	my $output = "";
        open(input,"temp");
        while(my $line = <input>){$output = $output.$line;}
        close(input);

        return  "\\begin{\\outputsize}\n" . "\\begin{Verbatim}\n" .
                $output .
                "\\end{Verbatim}\n" . "\\end{\\outputsize}\n";
}

\perlnewcommand{\printexec}[1]
{
	my $exec = $_[0];

	return "\\begin{\\outputsize}\n" . "\\begin{Verbatim}\n" .
		$exec . "\n" .
		"\\end{Verbatim}\n" . "\\end{\\outputsize}\n";
}

\perlnewcommand{\application}[1]
{
	my $app = $_[0];
	return "\\emph{" . $app . "}";
}

%\begin{document}

\index{RinDump!application writeup}
\section{\emph{RinDump}}
\subsection{Overview}
The application reads one or more RINEX (v.2+) observation files and dump the given observation IDs, linear combinations, satellite-dependent information or other things, to the screen, as a table, with one time and one satellite per line.

\subsection{Usage}
\subsubsection{\emph{RinDump}}

\begin{\outputsize}
\begin{longtable}{lll}
%\multicolumn{3}{c}{\application{RinDump}}\\
\multicolumn{3}{l}{\textbf{Required Arguments}}\\
\entry{Short Arg.}{Long Arg.}{Description}{1}
\entry{}{--file \<fn\>}{Name of file with more options [\#->EOL = comment]}{1}
& & \\
\multicolumn{3}{l}{\textbf{Optional Arguments}}\\
\entry{Short Arg.}{Long Arg.}{Description}{1}
\entry{}{--obs \<file\>}{Input RINEX observation file name [repeat]}{1}
\entry{}{--sat \<sat\>}{sat is a RINEX satellite id (see above) [repeat]}{1}
\entry{}{--dat \<data\>}{data (raw,combination, or other) to dump (see above) [repeat]).}{2}
\entry{}{--combo \<spec\>}{RINEX observation file; this option may be repeated.}{1}
\entry{}{--sys \<s\>}{System(s) (GNSSs) <s>=S[,S], where S=RINEX system [repeat] (GPS,GLO)RINEX systems are GPS,GLO,GAL,GEO or SBAS,COM
}{2}
\entry{}{--code \<s:c\>}{System \<s\> allowed tracking codes \<c\>, in order}{1}
\entry{}{--freq \<f\>}{Frequencies to use in solution [e.g. 1, 12, 5, 15] [repeat]}{1}
\entry{}{--eph \<fn\>}{Input Ephemeris+clock (SP3 format) file name(s) [repeat]}{1}
\entry{}{--nav <fn>}{Input RINEX nav file name(s) [GLO Nav includes freq channel] [repeat]}{2}
\entry{}{--obspath \<p\>}{Path of input RINEX observation file(s)}{1}
\entry{}{--ephpath \<p\>}{Path of input RINEX ephemeris+clock file(s)}{1}
\entry{}{--navpath \<p\>}{Path of input RINEX navigation file(s)}{1}
\entry{}{--start \<t[:f]\>}{Start processing data at this epoch ([Beginning of dataset])}{2}
\entry{}{--stop \<t[:f]\>}{Stop processing data at this epoch ([End of dataset])}{1}
\entry{}{--decimate \<dt\>}{Decimate data to time interval dt (0: no decimation)}{1}
\entry{}{--ref \<p[:f]\>}{Known position, default fmt f '\%x,\%y,\%z', for resids, elev, and ORDs}{2}
\entry{}{--GLOfreq \<sat:n\>}{GLO satellite and frequency channel number, e.g. R09:-7}{2}
\entry{}{--Trop \<m,T,P,H\>}{Trop model <m> [one of Zero,Black,Saas,NewB,Neill,GG,GGHt with optional weather T(C),P(mb),RH(\%)]}{3}
\entry{}{--timefmt \<fmt\>}{Format for time tags (see GPSTK::Epoch::printf) in output}{1}
\entry{}{--headless}{Turn off printing of headers (# ...) in output}{1}
\entry{}{--verbose}{Print extra output information}{1}
\entry{}{--debug}{Print debug output at level 0 [debug\<n\> for level n=1-7]}{1}
\entry{-h}{--help}{Print this syntax page, and quit}{1}
\entry{}{--typehelp}{Print all valid RINEX obs IDs, and quit{1}
\entry{}{--combohelp}{Print syntax for linear combination data tags, and quit}{1}
\end{longtable}

\begin{verbatim}
RinexDump usage: RinDump [-n] <rinex obs file> [<satellite(s)> <obstype(s)>] 

The optional argument -n tells RinexDump its output should be purely numeric.
\end{verbatim}
\end{\outputsize}

\subsection{Examples}
\begin{\outputsize}
\begin{lstlisting}
> RinDump algo1580.06o 3 4 5

# Rinexdump file: algo1580.06o Satellites: G03 G04 G05 Observations: ALL
# Week  GPS_sow Sat            L1 L S            L2 L S            C1 L S
1378 259200.000 G03  -3843024.647 0 3  -2994560.443 0 1  23796436.087 0 0
1378 259230.000 G03  -3954052.735 0 3  -3081075.654 0 2  23775308.750 0 0
1378 259260.000 G03  -4064994.465 0 2  -3167523.561 0 3  23754197.617 0 0

 . . .

          P2 L S            P1 L S            S1 L S            S2 L S
23796439.457 0 0  23796436.350 0 0        21.100 0 0        11.000 0 0
23775311.168 0 0  23775308.182 0 0        22.100 0 0        17.800 0 0
23754199.648 0 0  23754196.550 0 0        17.000 0 0        18.600 0 0

 . . .
\end{lstlisting}
\end{\outputsize}

\subsection{Notes}
MATLAB and Octave can read the purely numeric output.

%\end{document}
