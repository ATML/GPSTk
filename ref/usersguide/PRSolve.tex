%\documentclass{article}
%\usepackage{fancyvrb}
%\usepackage{perltex}
%\usepackage{xcolor}
%\usepackage{listings}
%\usepackage{longtable}
%\usepackage{multirow}
%\RecustomVerbatimEnvironment{Verbatim}{Verbatim}{frame=single}

\newcommand{\outputsize}{footnotesize}

\perlnewcommand{\getuse}[1]
{
        my $command = $_[0];
        $command = $command." > temp 2>&1";
        system("$command");

	my $output = "";
        open(input,"temp");
        while(my $line = <input>){$output = $output.$line;}
        close(input);

        return  "\\begin{\\outputsize}\n" . "\\begin{Verbatim}\n" .
                $output .
                "\\end{Verbatim}\n" . "\\end{\\outputsize}\n";
}

\perlnewcommand{\printexec}[1]
{
	my $exec = $_[0];

	return "\\begin{\\outputsize}\n" . "\\begin{Verbatim}\n" .
		$exec . "\n" .
		"\\end{Verbatim}\n" . "\\end{\\outputsize}\n";
}

\perlnewcommand{\application}[1]
{
	my $app = $_[0];
	return "\\emph{" . $app . "}";
}

%\begin{document}

\index{PRSolve!application writeup}
\section{\emph{PRSolve}}
\subsection{Overview}
The application reads one or more RINEX observation files, plus one or more
   navigation (ephemeris) files, and computes an autonomous pseudorange
   position solution, using a RAIM-like algorithm to eliminate outliers.
   Output is to the log file, and also optionally to a RINEX observation file with
   the position solutions in auxiliary header blocks.

\subsection{Usage}
\subsubsection{PRSolve}
\begin{\outputsize}
\begin{longtable}{lll}
%\multicolumn{3}{c}{\application{PRSolve}}\\
\multicolumn{3}{l}{\textbf{Required Arguments}} \\
\entry{Short Arg.}{Long Arg.}{Description}{1}
\entry{-o}{--obs}{Input RINEX observation file(s).}{1}
\entry{-n}{--nav}{Input navigation (ephemeris) file(s) (RINEX or SP3).}{1}
& & \\
\multicolumn{3}{l}{\textbf{Optional Arguments: Input}} \\
\entry{Short Arg.}{Long Arg.}{Description}{1}
\entry{-f}{}{File containing more options.}{1}
\entry{}{--obsdir}{Directory of input observation file(s).}{1}
\entry{}{--navdir}{Directory of input navigation file(s).}{1}
\entry{}{--metdir}{Directory of input meteorological file(s).}{1}
\entry{-m}{--met}{Input RINEX meteorological file(s).}{1}
\entry{}{--decimate}{Decimate data to time interval dt.}{1}
\entry{}{--BeginTime}{Start time: arg is `GPSweek,sow' OR `YYYY,MM,DD,HH,Min,Sec'.}{2}
\entry{}{--EndTime}{End time: arg is `GPSweek,sow' OR `YYYY,MM,DD,HH,Min,Sec'.}{2}
\entry{}{--useCA}{Use C/A code pseudorange if P1 is not available.}{1}
\entry{}{--forceCA}{Use C/A code pseudorange regardless of P1 availability.}{2}
& & \\
\multicolumn{3}{l}{\textbf{Optional Arguments: Configuration}} \\
\entry{Short Arg.}{Long Arg.}{Description}{1}
\entry{}{--Freq}{Frequency to process: 1, 2, or 3 for L1, L2, or iono-free combination.}{2}
\entry{}{--MinElev}{Minimum elevation angle in degrees (only if --PosXYZ).}{2}
\entry{}{--exSat}{Exclude this satellite.}{1}
\entry{}{--Trop}{Trop model, one of ZR, BL, SA, NB, NL, GG, GGH (gpstk::TropModel), with optional weather T(c), P(mb),RH(\%).}{3}
& & \\
\multicolumn{3}{l}{\textbf{Optional Arguments: PRSolution Configuration}} \\
\entry{Short Arg.}{Long Arg.}{Description}{1}
\entry{}{--RMSlimit}{Upper limit on RMS post-fit residuals (m) for a good solution.}{2}
\entry{}{--SlopeLimit}{Upper limit on RAIM `slope' for a good solution.}{1}
\entry{}{--Algebra}{Use algebraic algorithm (otherwise linearized LS).}{2}
\entry{}{--DistanceCriterion}{Use distance from a priori as convergence criterion (else RMS).}{2}
\entry{}{--ReturnAtOnce}{Return as soon as a good solution is found.}{1}
\entry{}{--NReject}{Maximum number of satellites to reject.}{1}
\entry{}{--NIter}{Maximum iteration count (linearized LS algorithm).}{2}
\entry{}{--Conv}{Minimum convergence criterion (m) (LLS algorithm).}{2}
& & \\
\multicolumn{3}{l}{\textbf{Optional Arguments: Output}} \\
\entry{Short Arg.}{Long Arg.}{Description}{1}
\entry{}{--Log}{Output log file name (prs.log).}{1}
\entry{}{--PosXYZ $<$X,Y,Z$>$}{Known position (ECEF,m), used to compute output residuals.}{2}
\entry{}{--APSout}{Output autonomous pseudorange solution (APS - no RAIM).}{2}
\entry{}{--TimeFormat}{Output time format (ala CommonTime) (default: \%4F \%10.3g).}{2}
& & \\
\multicolumn{3}{l}{\textbf{Optional Arguments: RINEX Output}} \\
\entry{Short Arg.}{Long Arg.}{Description}{1}
\entry{}{--outRinex}{Output RINEX observation file name.}{1}
\entry{}{--RunBy}{Output RINEX header `RUN BY' string.}{1}
\entry{}{--Observer}{Output RINEX header `OBSERVER' string.}{1}
\entry{}{--Agency}{Output RINEX header `AGENCY' string.}{1}
\entry{}{--Marker}{Output RINEX header `MARKER' string.}{1}
\entry{}{--Number}{Output RINEX header `NUMBER' string.}{1}
& & \\
\multicolumn{3}{l}{\textbf{Optional Arguments: Help}} \\
\entry{Short Arg.}{Long Arg.}{Description}{1}
\entry{}{--verbose}{Print extended output.}{1}
\entry{}{--debug}{Print very extended output.}{1}
\entry{}{--helpRetCodes}{Print return codes (implies --help).}{1}
\entry{-h}{--help}{Print syntax and quit.}{1}
\end{longtable}
\end{\outputsize}

\subsection{Examples}
\begin{\outputsize}
\begin{verbatim}
> PRSolve -o arl2800.06o -n arl2800.06n

PRSolve, part of the GPS ToolKit, Ver 2.3 11/09, Run 2011/07/22 11:39:15
Opened log file prs.log

Weighted average RAIM solution for file: arl2800.06o
 (2880 total epochs, with 2880 good, 0 rejected.)
    918129.266960  -4346070.850055   4561977.615781
Covariance of RAIM solution for file: arl2800.06o
         0.000150        -0.000061         0.000058
        -0.000061         0.000427        -0.000248
         0.000058        -0.000248         0.000493
\end{verbatim}
\end{\outputsize}
\subsection{Notes}
 In the log file, results appear one epoch per line with the format:\\
 TAG Nrej week sow Nsat X Y Z T RMS slope nit conv sat sat .. (code) [N]V\\
 TAG denotes solution (X Y Z T) type:\\
\begin{itemize}
\item  RPF  Final RAIM ECEF XYZ solution
\item RPR  Final RAIM ECEF XYZ solution residuals [only if --PosXYZ given]
\item RNE  Final RAIM North-East-Up solution residuals [only if --PosXYZ]
\item  APS  Autonomous ECEF XYZ solution [only if --APSout given]
\item APR  Autonomous ECEF XYZ solution residuals [only if both --APS \& --Pos] 
\item ANE  Autonomous North-East-Up solution residuals [only if --APS \& --Pos]
\end{itemize}
Where:\\
\begin{itemize}
\item Nrej = number of rejected sats
\item (week,sow) = GPS time tag
\item Nsat = \# sats used
\item XYZT = position+time solution(or residuals)
\item RMS = RMS residual of fit
\item slope = RAIM slope
\item nit = \# of iterations
\item conv = convergence factor
\item `sat sat ...' lists all sat. PRNs (- : rejected)
\item code = return value from PRSolution::RAIMCompute()
\item NV means NOT valid
\end{itemize}
%\end{document}

