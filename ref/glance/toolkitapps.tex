\documentclass{article}
\usepackage{multirow}
\usepackage{rotating}
\usepackage{colortbl}
\usepackage{xcolor}
\usepackage[total={8.5in,11in},top=2cm,left=1.5cm,landscape=true]{geometry}

\newcommand{\sideco}{gray}
\newcommand{\sideso}{1.0}
\newcommand{\sidect}{gray}
\newcommand{\sidest}{1.0}
\newcommand{\sidewidth}{0.15in}
\newcommand{\entco}{gray}
\newcommand{\entso}{1.00}
\newcommand{\entct}{gray}
\newcommand{\entst}{0.95}

\newcommand{\twidth}{2.0in}
\newcommand{\dwidth}{3.10in}
\newcommand{\ewidth}{3.10in}

\newcommand{\appdesce}[3] {\cellcolor[\sideco]{\sideso} & \cellcolor[\entct]{\entst} \multirow{1}{\twidth}{#1} & \cellcolor[\entct]{\entst} \multirow{1}{\dwidth}{\footnotesize{#2}} & \cellcolor[\entct]{\entst} \multirow{1}{\ewidth}{\ttfamily{\footnotesize{#3}}}}

\newcommand{\appdesco}[3]{\cellcolor[\sideco]{\sideso} & \cellcolor[\entco]{\entso} \multirow{1}{\twidth}{#1} & \cellcolor[\entco]{\entso} \multirow{1}{\dwidth}{\footnotesize{#2}} & \cellcolor[\entco]{\entso} \multirow{1}{\ewidth}{\ttfamily{\footnotesize{#3}}}}

\pagestyle{empty}

\begin{document}

\begin{tabular}{clll}

\multicolumn{4}{c}{GPSTk Apps at a Glance} \\
& \textbf{Tool} & \textbf{Description} & \textbf{Execution Example} \\
\hline

\appdesco{calgps}
	 {generates a GPS calendar}
	 {calgps -Y 2004} \\

\appdesce{poscvt}
	 {converts a given input position to other position formats}
	 {poscvt --geodetic="30.28 262.26700 167.64" } \\

\appdesco{timeconvert}
         {converts given input time to other time formats}
         {timeconvert --calendar="07 04 2006"} \\ 

\appdesce{wheresat}
         {outputs expected location of a satellite}
	 {wheresat -b arl2100.06n -p 3} \\
\hline

\multirow{-5}{\sidewidth}{\rotatebox{90}{\tiny{\hspace{3mm}Transforms}}} 

\appdesco{ash2mdp}
         {converts Ashtech Z(Y)-12 data to MDP}
	 {TC or JL} \\

\appdesce{rtAshtech}
         {records observations from an Ashtech receiver}
         {rtAshtech -p /dev/ttyS1 -o "minute\%03j\%02H\%02m.\%06yo"} \\

\appdesco{ficfica/ficafic/fic2rin}
         {convert FIC files between ASCII, binary, and RINEX formats}
	 {fic2rin fic2100.06 rin121.06n} \\

\appdesce{mdp2fic/mdp2rinex}
         {convert MDP files to FIC or RINEX files}
	 {mdp2rinex -i mdpfile -o arl2100.06o} \\

\appdesco{novaRinex}
         {convert Novatel files to RINEX files}
         {novaRinex --input nova2100.06 --obstype L1} \\

\appdesce{navdmp}
         {dumps information from nav files to human readable formats}
         {navdmp -i arl2100.06n -o arl2100.06.dmp} \\

\appdesco{RinexDump}
         {flattens a RINEX file into a matrix}
	 {RinexDump arl2100.06o 3 4 L1 L2} \\
\hline 

\multirow{-8}{\sidewidth}{\rotatebox{90}{\tiny{\hspace{2mm} Collecting/Converting}}}

\appdesce{daa}
         {performs a data availability analysis of collected data}
	 {TC or JL} \\

\appdesco{ddGen}
         {computes double-differences from raw observations}
	 {TC or JL} \\

\appdesce{ephdiff}
         {compares the satellite positions from two ephemeris sources}
	 {ephdiff arl2100.06n fic2100.06} \\

\appdesco{ficdiff}
         {compares contents of two FIC files}
	 {ficidff fic12100.06 fic22100.06} \\

\appdesce{ficcheck/ficacheck}
         {reads a FIC file and checks it for errors reporting the first found}
	 {ficcheck fic2100.06 -t "07/20/2006 11:00:00"} \\

\appdesco{findMoreThan12}
         {predicts when more than twelve satellites are in view}
	 {findMoreThan12 -e arl1256.06n -p "X Y Z" -m 10} \\

\appdesce{mdptool}
         {manipulates MDP data streams}
	 {mdptool -i mdpfile --pvt --obs} \\

\appdesco{mpsolve}
         {computes statistical model of a dual frequency multipath combination}
	 {mpsolve -o arl2100.06o -e arl2100.06n} \\

\appdesce{ordGen/Clock/LinEst/Edit/Stats}
         {models observed range deviations}
	 {TC or JCL} \\

\appdesco{RinSum}
         {summarizes contents of a RINEX observation file}
	 {RinSum -i arl2100.06o --EpochBeg 2006,07,20,13,20,00} \\ 

\appdesce{rmwdiff/rnwdiff/rowdiff}
         {compares contents of two RINEX files}
	 {rowdiff arl1210.06o arl22100.06o} \\

\appdesco{row/rnw/rmwcheck}
         {read RINEX files and checks it for errors reporting the first found}
	 {rnwcheck arl210.06n -e "07/20/2006 11:00:00"} \\

\appdesce{rstats}
         {computes standard, robust statistics on numbers in text stream}
	 {BT} \\
\hline


\multirow{-11}{\sidewidth}{\rotatebox{90}{\tiny{Comparing \& Validating}}}

\appdesco{DiscFix}
         {cycle slip corrector}
	 {DiscFix -i arl2100.06o --DT 1.5} \\

\appdesce{EditRinex}
         {edits a RINEX observations file based on user input}
	 {BT} \\

\appdesco{mergeFIC}
         {sorts and merges input FIC files into a single file}
	 {mergeFIC -i fic12100.06 -i fic22100.06 -o ficmerge2100.06} \\

\appdesce{mergeRinObs/Nav/Met}
         {sorts and merges RINEX files}
	 {mergeRinNav -i arl2100.06n -i arl2110.06n arl210-211.06n} \\

\appdesco{NavMerge}
         {merges RINEX nav files into a single file}
	 {NavMerge -oarlnavs.06n arl2100.06n arl2110.06n} \\

\appdesce{rinexthin}
         {decimates an input RINEX observation files to desired data rate}
	 {rinexthin -f arl2100.06o -s 30 -o arl2100thin.06n} \\

\appdesco{ResCor}
         {edits RINEX files and computes corrections}
	 {ResCor -IFalr2100.06o -OFarl2100mod.06o -DS12,12:00:00} \\

\appdesce{sp3version}
         {read and write from and to either a or c format SP3 file}
	 {JV} \\ \hline

\multirow{-8}{\sidewidth}{\rotatebox{90}{\tiny{Editing Data}}}

\appdesco{IonoBias}
         {solves interfrequency biases and a simple ionosphere model}
	 {IonoBias --input arl2100.06o --nav arl2100.06n --XSat 3} \\

\appdesce{TECMaps}
         {creates maps of Total Electron Content (TEC)}
	 {TECMaps --input arl2100.06o --nav arl2100.06n --LinearFit} \\
\hline

\multirow{-3}{\sidewidth}{\rotatebox{90}{\tiny{\hspace{3mm}Iono}}}

\appdesco{DDBase}
         {computes a network solution using carrier phase}
	 {DDBase ... --ObsFile arl2100.06o --PosXYZ x,y,z,1 --Fix ...} \\

\appdesce{ddmerge}
         {appends elevation and azimuth data taken from the RAW file to the appropriate line in the DDR file and output to the output file}
	 {BT} \\

\appdesco{mergeSRI}
         {combines solution and covariance results from different sources into a single result}
	 {BT} \\

\appdesce{PosInterp}
         {calculates user positions at a higher rate than observations}
	 {BT} \\

\appdesco{PRSolve}
         {generates autonomous position solution}
	 {PRSolve -o alr2100.06o -n arl2100.06nn --XPRN 12} \\

\appdesce{rinexpvt}
         {generates autonous position solution}
	 {rinexpvt -o alr2100.06o -n arl2100.06n} \\

\appdesco{vecsol}
         {estimates short baseline using range or carrier phase}
	 {vecsol station12100.06o station22100.06o}  \\

\hline

\multirow{-7}{\sidewidth}{\rotatebox{90}{\tiny{\hspace{3mm}Positioning}}}


\end{tabular}

\end{document}
